
% !TeX spellcheck = de_DE
\documentclass{article}

\usepackage[ngerman]{babel}
\usepackage{graphicx}
\usepackage{indentfirst}
\usepackage{hyperref}
\usepackage{geometry}
\usepackage{changepage}

\graphicspath{ {./images/} }

\makeatletter
\newcommand{\sectionauthor}[1]{
	{\parindent 0em \large \scshape Autor: #1 \par \nobreak \vspace*{1em}}
	\@afterheading
}
\newcommand{\specification}[3]{
	{\parindent 0.5em \hangindent 3em \hypertarget{spec:#1:#2}{\textbf{/#1#2/}} #3 \par \nobreak \vspace*{0.5em}}
}
\makeatother
\begin{document}
%--Einleitung--------------------------------------------------------------------------------------------------------------------------------------------------------------------------
\section{Einleitung}
\sectionauthor{Jonas Picker}

%--Systemarchitektur---------------------------------------------------------------------------------------------------------------------------------------------------------------
\section{Systemarchitektur}
\sectionauthor{Ivan Charviakou}

%--Klassendiagramm---------------------------------------------------------------------------------------------------------------------------------------------------------------
\section{Klassendiagramm}
\sectionauthor{Mohamad Najjar}

%--JSF-Dialoge-----------------------------------------------------------------------------------------------------------------------------------------------------------------------
\section{JSF-Dialoge}
\sectionauthor{León Liehr}

%--Systemfunktionen----------------------------------------------------------------------------------------------------------------------------------------------------------------
\section{Systemfunktionen}
\sectionauthor{Jonas Picker}

%--Datenfluss--------------------------------------------------------------------------------------------------------------------------------------------------------------------------
\section{Datenfluss}
\sectionauthor{Sergei Pravdin}
Die Kommunikationen zwischen den Klassen und die Interaktionen des Systems werden durch den Sequenzdiagramme abgebildet. Um einen Datenfluss beispielhaft zu zeigen, werden die zwei Szenarien vorgelegt. Zuerst bucht ein angemeldeter Nutzer ein Medium-Exemplar erfolgreich zur Ausleihe. Im zweiten Szenario bucht ein angemeldeter Nutzer ein Medium-Exemplar erfolglos zur Ausleihe, weil die Verbindung mit der Datenbank fehlgeschlagen ist.
\subsection{Interaktionen bei einem erfolgreichen Buchen eines Medium-Exemplars}
Der Nutzer befindet sich auf der Mediensuche-Seite und wünscht das Buch 'Programmieren lernen' zu buchen. Im System existiert das Medium mit dem Titel 'Programmieren lernen' und der Signatur '17RE'. Ein Exemplar mit der Signatur '17RE (+1)' gehört zu dem genannten Medium. Das System ist so eingestellt, dass die angemeldeten Nutzer Zugriff auf den Medien haben.
\subsubsection{Suchen nach dem Medium}
Der Nutzer gibt 'Programmieren lernen' und '17RE' in die Suchfelder 'Titel' und 'Signatur' und klick auf den Suchen-Button. Das Mediensuche-Bean kapselt die Sucheingabe ins MediensucheDTO ein. Das Mediensuche-DTO ruft die Methode 'getMedien' aus dem Mediensuche-DAO auf. Das Mediensuche-DAO ruft die Methode 'getConnection' aus dem Connection-Pool-Bean und bekommt eine Connection von dem zurück. Danach führt das Mediensuche-DAO eine selectSQL-Anfrage durch und gibt eine Liste der entsprechenden Medien dem Mediensuche-DTO zurück. Im nächsten Schritt leitet Mediensuche-DTO diese Liste der Medien dem Mediensuche-Bean weiter. Das Mediensuche-Bean ruft seine Methode 'updateMedien' auf und gibt dem Nutzer die aktualisierte Mediensuche-Seite zurück.
\subsubsection{Navigation zum Medium}
Der Nutzer klickt auf das angezeigte Medium 'Programmieren lernen'. Das Mediensuche-Bean ruft die Methode 'navigate' auf und gibt die Mediumsansicht-Seite zurück.
\subsubsection{Buchen eines Exemplares}
Der Nutzer klickt auf den Buchen-Button. Das Medium-Bean ruft die Methode checkStatus aus dem UserSession-Bean auf, um zu prüfen, ob der Nutzer Zugriff zum Buchen hat. Das UserSession-Bean gibt das positive Ergebnis dem Medium-Bean zurück. Das Medium-Bean kapselt das Buchen ins Medium-DTO ein und das Medium-DTO ruft die Methode book() aus dem Medium-DAO auf. Das Medium-DAO ruft die Methode 'getConnection' aus dem Connection-Pool-Bean und bekommt eine Connection von dem zurück. Danach führt das Medium-DAO eine updateSQL-Anfrage durch. Im nächsten Schritt gibt Medium-DTO  dem Medium-Bean das Ergebnis der Operation zurück. Das Medium-Bean ruft seine Methode 'updateExamples' auf und gibt dem Nutzer die aktualisierte Medium-Seite zurück. Das Exemplar ist erfolgreich gebucht.


%--ER-Modell--------------------------------------------------------------------------------------------------------------------------------------------------------------------------
\section{ER-Modell}
\sectionauthor{Jonas Picker}


