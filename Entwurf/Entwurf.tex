
% !TeX spellcheck = de_DE
\documentclass{article}

\usepackage[ngerman]{babel}
\usepackage{graphicx}
\usepackage{indentfirst}
\usepackage{hyperref}
\usepackage{geometry}
\usepackage{changepage}

\graphicspath{ {./images/} }

\makeatletter
\newcommand{\sectionauthor}[1]{
	{\parindent 0em \large \scshape Autor: #1 \par \nobreak \vspace*{1em}}
	\@afterheading
}
\newcommand{\specification}[3]{
	{\parindent 0.5em \hangindent 3em \hypertarget{spec:#1:#2}{\textbf{/#1#2/}} #3 \par \nobreak \vspace*{0.5em}}
}
\makeatother

\begin{document}

%--Einleitung--------------------------------------------------------------------------------------------------------------------------------------------------------------------------
\section{Einleitung}
\sectionauthor{Jonas Picker}
Hier wird der Entwurf des Bibliotheksmanagementsystems BiBi beschrieben. Das Dokument bezieht sich auf das Lastenheft von Christian Bachmaier und Armin Größlinger und stellt eine technische Vertiefung unseres Pflichtenhefts dar, welches im Folgenden wiederholt referenziert wird. 

%--Systemarchitektur---------------------------------------------------------------------------------------------------------------------------------------------------------------
\section{Systemarchitektur}
\sectionauthor{Ivan Charviakou}

\subsection{Design-Patterns: Schichtenunabhängigkeit}

MVC: Als grundlegendes Design-Pattern sorgt MVC dafür, dass die Model-, Präsentations-, und Steuerungskomponente voneinander weitgehend entkoppelt sind. Dabei implementiert das JSF-Framework per Default Großteile der Präsentations- und Steuerungskomponenten, aber überlässt die Modelkomponente komplett dem Entwickler. Für die Präsentationskomponente werden nämlich vordefinierte UI-Komponente vorgesehen, während der eingebaute Faces-Servlet die Steuerungsaufgabe übernimmt. Durch selbst-definierte Tag-Libraries und Facelets lassen sich diese beiden Komponenten auch ggf. vervollständigen. \vspace{0.5em}

CDI: Die im JSF-Framework eingebaute CDI-Funktionalität ermöglicht eine direkte Kommunikation zwischen abhängigen Komponenten, die durch JSF verwaltet werden. Solche Beziehungen werden dann mit entsprechenden Annotationen, wie ‚@inject‘, gekennzeichnet. In der gegebenen Anwendung trägt CDI dadurch wesentlich zur Intraschichtenkommunikation bei. \vspace{0.5em}

DTOs: Ein Datentransferobjekt (DTO) enthält die Daten zu einer bestimmten Entität oder zu einem zusammenhängenden Ausschnitt aus verschiedenen Entitäten. In der gegebenen Anwendung entspricht das einem Medium, Exemplar, Nutzer, usw. Durch Übergabe von solchen Objekten an verschiedenen Modelschichten, können sie die Daten unabhängig voneinander anpassen oder evaluieren. Als Beispiel prüft im Model die Logikschicht das Format eines E-Mails, während die Sicherheitsschicht SQL-Injection-Attacken in der Eingabe ausschließt. \vspace{0.5em}

DAOs und ihre Fassaden: Zu jedem DTO existiert auch ein Datenzugriffsobjekt (DAO), das bei einer Übergabe eines DTOs die enthaltenen Daten speichert bzw. lädt. Damit mehrere Modelschichten diese Prozeduren eigens erweitern können, verwenden die DTOs in der gegebenen Anwendung den Fassade-Muster. Jede betroffene Schicht verwendet zur Erweiterung eines DAOs nämlich die gleiche Schnittstelle und ruft die Schnittstelle der nächsten Schicht auf. Als Beispiel gibt die Logikschicht ein Passwort im Klartext an und übergibt es an eine DAO-Instanz in der Sicherheitsschicht mittels eines DTOs. Diese Schicht berechnet dann aus dem Klartext einen Hash-Wert und leitet diesen Wert an die DAO-Instanz in der Datenzugriffsschicht weiter. \vspace{0.5em}

Event-Listener: Dadurch, dass JSF verschiedene Arten von Listenern bereitstellt, ist eine stärkere Entkopplung von Schichten möglich. Während Action-, Value-Change-, und Phasenlistener hauptsächlich eine Logikschicht ansprechen, kann beispielweise der eingebaute System-Event-Listener für die unabhängige Initialisierung der Modelschichten sorgen.

\subsection{Design-Patterns: Fehlerbehandlung}

Allgemeiner Testmodus: Es wird ein Testmodus eingeführt, der beim Applikationsstart als Parameter angegeben wird. \vspace{0.5em}

Exceptions: Jede Schicht in der gegebenen Anwendung definiert eigene Checked-Exceptions, die ggf. durch die Schichten hochgeworfen werden. Dabei wird eine solche Exception in jeder Zwischenschicht gefangen in eine eigens definierte Exception umgewandelt bis eine Schicht sie behandelt. Im Gegensatz werden Unchecked-Exceptions nicht behandelt und führen zum Absturz der Anwendung. Während für eine Präsentations- und Logikschicht oft eine GUI-Anzeige als geeignete Behandlung gilt, ist es für jede Schichten anders. Falls es beispielsweise durch eine Checked-Exception in der Datenzugriffsschicht festgestellt wird, dass die verwendete Datenbank nicht mehr erreichbar ist, wird stattdessen dynamisch auf lokale Applikationsdaten zugegriffen. Unter den DAOs in der Datenzugriffsschicht wird somit der Strategy-Muster angewendet. \vspace{0.5em}

Logging: Als Logging-Framework wird der im JDK vordefinierte Framework verwendet. Diese Funktionalität ist nur im Testmodus aktiv und es wird zu einer lokalen Log-Datei geschrieben. \vspace{0.5em}

Validatoren: Das JSF-Framework gibt dem Entwickler die Möglichkeit, Daten auf Korrektheit zu prüfen, bevor sie dem Logikschicht bereitgestellt wird. Da aber komplexere semantische Fehler oft einen größeren Kontext und eine komplexere Behandlung benötigen, wird der Einsatz von diesen Validatoren hauptsächlich auf syntaktische Formattierfehler beschränkt. 

\subsection{Design-Patterns: Konkrete Features}

Connection-Pool: Unter einem Pool bezeichnet man eine Sammlung an Objekten, die angefragt, zurückgegeben, und wiederverwendet werden. Dies vermeidet insbesondere die Erzeugung von solchen Objekten, was in Bezug auf Datenbankverbindungen als aufwendig gilt. Da es trivialerweise nur einen solchen Container geben sollte, wird ein Pool auch als Singleton verwendet. Für die gegebene Anwendung wird die HikariCP Implementierung einer solchen Connection-Pool genommen. Beim Testmodus wird die Anzahl an Verbindungen auf eins gesetzt. \vspace{0.5em}

Wartungsthread: In der gegebenen Anwendung kann es vorkommen, dass persistierte Daten oder Daten im Cache mit der Zeit ungültig werden. Du solchen Daten gehört beispielsweise die Gültigkeit von einem Passwortrücksetzungslink. Mit einem Wartungsprozess werden solche Inkonsistenzen erkannt und behoben.

\subsection{Modeldiagramm}

\begin{figure}[h]
	\centering
	\includegraphics[width = 50em]{Modeldiagramm}
\end{figure}

%--Klassendiagramm---------------------------------------------------------------------------------------------------------------------------------------------------------------
\section{Klassendiagramm}
\sectionauthor{Mohamad Najjar}

%--JSF-Dialoge-----------------------------------------------------------------------------------------------------------------------------------------------------------------------
\section{JSF-Dialoge}
\sectionauthor{León Liehr}

%--Systemfunktionen----------------------------------------------------------------------------------------------------------------------------------------------------------------
\section{Systemfunktionen}
\sectionauthor{Jonas Picker}
\subsection{Technische Systemsicherheit} +Session-Hijacking
\noindent \textbf{Kommunikationsverschlüsselung:} Durch das vorrausgesetzte SSL-Zertifikat des Tomcat-Servers wird der Anwendung die Kommunikation mit dem HTTPS-Protokoll ermöglicht. Dies ist eine Ergänzung der HTTP-Kommunikation um eine Transportverschlüsselung. Zusätzlich zur Vereitelung von Abhörversuchen der an Ihren Server gesendeten Anfragen signalisieren Sie den Klienten so auch die Vertrauenswürdigkeit Ihrer Institution durch die CA-Zertifizierung.\\
\textbf{Nutzerberechtigungen:} Um zu Verhindern, dass Nutzer ohne ausreichende Berechtigungen auf zugangsbeschränkte Bereiche des Webspaces zugreifen können, werden entsprechende Maßnahmen ergriffen: Um den direkten Aufruf von Seiten, die für den Nutzer nicht zugänglich sein sollten, zu unterbinden, verwenden wir einen sogenannten Phaselistener. Wie jede JSF-Anwendung durchläuft auch unsere nach Erhalt einer Anfrage eine Reihe von Verarbeitungsphasen, bevor die Antwort an den Klienten fertiggestellt und verschickt wird. Ein Phaselistener kann während dieses Prozesses zusätzliche Integritätsbedingungen abprüfen und gegebenenfalls die Antwort auf die Anfrage verändern/verhindern. Zusätzlich wird durch das in JSF eingebaute Session-Tracking\footnote{Das Verfolgen der Klientenverbindung zwischen den eigentlich verbindungslosen HTTP-Anfragen} sichergestellt, dass die verschiedenen Nutzerrollen eine unterschiedliche Version der gleichen Seiten angezeigt bekommen, deren Funktionalitäten den Berechtigungen der Rollen entsprechen. \\
\textbf{Vorbeugen von Angriffen:} Durch geschicktes Einschleusen von SQL\footnote{Structured Query Language, eine weit verbreitete Datenbankanfragesprache}-Code in ungesicherte Formularfelder könnten Unbefugte direkten Zugriff auf die darunterliegende Datenbank erhalten. Der Datenbankzugriff in unserem System erfolgt via JDBC\footnote{Java Database Connectivity, eine universelle Datenbankschnittstelle}, welche bereits über einen Mechanismus zur Verhinderung von diesen SQL-Injections verfügt. Hierbei wird verhindert, dass von Nutzern eingegebener Text vom Managementsystem der Datenbank interpretiert wird. Ein weiterer Angriffsvektor stellt das sogenannte Cross-Site-Scripting dar. Dabei wird versucht, nutzergenerierte Teile von Websiten mit vom Browser interpretierbaren Codeabschnitten zu füllen, welche dann beim Anzeigen des Inhalts bei anderen Klienten ausgeführt werden. Es gibt verschiedene Versionen dieser Angriffsart, von denen nicht alle am Server verhindert werden können. Jedoch wird in der Anwendungslogik jede Nutzereingabe auf interpretierbaren HTML (und eingebundenen JavaScript) Code untersucht und gegebenenfalls entschärft.
\subsection{Fehlerbehandlung und Logging}
Von uns vorhergesehene Fehler und einige im Hintergrund ablaufende Prozesse werden durch einen selbst erstellten Logging-Mechanismus dokumentiert. Da dieser Mechanismus während des Entwicklungsprozesses ebenfalls nützlich ist und auf feinster Einstellung für den Anwender uninteressant sein könnte, wird es eine Option geben, die den Detailgrad und die Anzahl der ausgegebenen Meldungen festlegt. Alle Meldungen werden in eine seperate Log-Datei geschrieben und können wahlweise auf der Konsole angezeigt werden. Vorhergesehene Fehler in Abläufen werden von der Anwendung bestmöglich abgehandelt und wie erwähnt dokumentiert. Sollte jedoch ein nicht vorhergesehener Fehler auftreten, wird der Grund dafür in From der Java-eigenen Exceptions auf die Konsole ausgegeben. Durch umfangreiches Testen der Anwendung vor Auslieferung versuchen wir, Sie vor einem solchen Fall zu bewahren.
\subsection{Selbstständige Prozesse}
Im System gibt es verschiedene Fristen (Abholungsmarkierung, Ausleihrückgabe [global, pro Medium, pro Nutzer], Gültigkeit des Verifizierungs-/Zurücksetzungslinks, Mahnungszeitversatz) und damit verbundene Aktionen. Um diese Aktionen ausführen zu können, muss das System in gewissen Abständen selbstständig auf die Datenbank zugreifen. In der Klasse ???? ist dafür ein eigener Wartungsthread implementiert.
\subsection{Starten und Stoppen der Anwendung}
\noindent \textbf{Start:} Beim Systemstart wird zunächst der oben beschriebene Logging-Mechanismus initialisiert. Im Anschluss liest das System die Konfigurationsdatei ein und versucht die dort angegebene Verbindung zur Datenbank. Bei erfolgreicher Verbindung wird überprüft, ob die benötigten Tabellennamen alle existieren bzw. legt diese bei deren Fehlen an. Sollte die Struktur der Tabellen fehlerhaft sein, wird eine Fehlermeldung geloggt und die Tabellen werden nicht vom System benutzt. Wenn die Datenbank fehlerhaft/nicht verbunden ist, werden eingehende Anfragen mit einer Fehlerseite beantwortet. Nach erfolgreichem Ablauf der vorhin genannten Prozesse wird als nächstes der Wartungsthread angestoßen, um fristenabhängige Aktionen durchzuführen. Danach ist das System vollständig betriebsbereit.\\
\textbf{Stop:} 
Beim planmäßigen Herunterfahren des Systems werden zunächst die noch offenen Datenbankverbindungen geschlossen und danach die Anwendung gestoppt. Dda dies sehr schnell passiert, wird währenddessen auf eine extra Umleitung der Nutzer auf eine Fehlerseite verzichtet. Da unsere SQL-Transaktionen dem ACID\footnote{Atomicity, Consistency, Isolation, Durability}-Prinzip folgen, wird die Datenbank jedoch auch beim Herunterfahren während eines Zugriffs in konsistentem Zustand hinterlassen. Nutzersessions überleben das Ausschalten des Servers nicht, d.h. alle eingeloggten Nutzer sind nach einem Neustart des Systems ausgeloggt. Wenn die Java Laufzeitumgebung des Systems plötzlich beendet wird und das System abstürzt, wird mittels einer sogenannten ShutdownHook trotzdem noch versucht, offene Datenbankverbindungen zu schließen. Dies ist bei einem Stromausfall allerdings unmöglich. 
%--Datenfluss--------------------------------------------------------------------------------------------------------------------------------------------------------------------------
\section{Datenfluss}
\sectionauthor{Sergei Pravdin}

%--ER-Modell--------------------------------------------------------------------------------------------------------------------------------------------------------------------------
\section{ER-Modell}
\sectionauthor{Jonas Picker}

\subsection{Legende}
Jede Entität wird im Folgenden zusammen mit ihren nicht offensichtlichen Attributen und Relationen kurz beschrieben.\\
\textbf{Medium:} Diese Entität modelliert die in der Bibliothek gehaltenen Medien. Zusätzlich zum Primärschlüssel ist auch das Attribut 'Rückgabefrist' und 'Icon' nicht vom Administrator entfernbar, die restlichen Attribute (kursiv) stellen den modifizierbaren Standartsatz der Medienattribute dar (PfHft /D020/). Erwähnenswert ist hier noch das als mehrwertig markierte Attribut 'Autoren', diese Markierung kann auch zu benutzerdefinierten Attributen hinzugefügt werden (PfHft. /F380/).\\
\textbf{Kategorie:} Die mit dieser Entität verbundenen Relationen ordnen jedem Medium maximal eine Kategorie zu und modellieren die Kategoriehierarchie (PfHft. /W440/) durch die Selbstbeziehung.\\
\textbf{Exemplar:} Von jedem Medium muss mindestens ein Exemplar vorhanden sein. Auch kann ein bestimmtes Exemplar von genau einem Nutzer zur Abholung markiert werden, diese Aktion (genau wie die Abholung oder Rückgabe) ändert den Verfügbarkeitsstatus und die dazugehörige Operationsfrist dementsprechend. \\
\textbf{Benutzer:} Ob ein Benutzer die Ausleihfunktion benutzen kann, wird durch das Attribut 'Ausleihstatus' modelliert. 'Accountstatus' zeigt hingegen an, ob der Nutzeraccount bereits den Verifizierungsprozess durchlaufen hat (PfHft. /W70/). Das Passwort wird in gehashter Form abgespeichert. \\
\textbf{Linkgenerator:} Diese Entität kapselt einen befristet gültigen URL-Suffix aus dem der Verifizierungs- oder Passwortzurücksetzungslink (je nach Zweck) für genau einen Account erstellt wird.\\
\textbf{Anwendung:} Hier werden die setzbaren globalen Variablen und Anwendungseinstellungen gespeichert. Da die Tabelle nur einen Eintrag hat, wird auf einen Primärschlüssel verzichtet. 'Anonymer Zugang' modelliert die Berechtigungen anonymer Nutzer beim Besuchen des Webspaces (PfHft. /F10/), während 'Registrierungsstatus' eine nutzerfreundlichere Alternative als den RegEx zum Sperren der Registrierung bietet (PfHft. /F20/). 'Ausleihstatus' steht für das Umschalten des Systems zur manuellen Freischaltung der Ausleihfunktion für registrierter Nutzer. \\

\end{document}

