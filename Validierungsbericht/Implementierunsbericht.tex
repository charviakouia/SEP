
% !TeX spellcheck = de_DE
\documentclass{article}

\usepackage[ngerman]{babel}
\usepackage{graphicx}
\usepackage{float}
\usepackage{booktabs}
\usepackage{lscape}
\usepackage{longtable}
\usepackage{geometry}
\usepackage{caption}
\usepackage{subcaption}

\graphicspath{ {./images/} }
\setlength\parindent{0pt}

\setlength\LTleft{0pt}
\setlength\LTright{0pt}

\makeatletter
\newcommand{\sectionauthor}[1]{
	{\parindent 0em \large \scshape Autor: #1 \par \nobreak \vspace*{1em}}
	\@afterheading
}
\newcommand{\specification}[3]{
	{\parindent 0.5em \hangindent 3em \hypertarget{spec:#1:#2}{\textbf{/#1#2/}} #3 \par \nobreak \vspace*{0.5em}}
}
\makeatother

\title{BiBi - Validierungsbericht}
\date{\today\\v1.0}
\author{
	Ivan Charviakou\\
	León Liehr\\
	Jonas Picker\\
	Sergei Pravdin
}

\begin{document}

%--Titel----------------------------------------------------------------------------------------------------------------------------------------------------------------------------------
\maketitle
\begin{figure}[H]
	\centering
	\includegraphics[width = 30em]{Logo}
\end{figure}
\newpage
\tableofcontents
\newpage

%--Einleitung--------------------------------------------------------------------------------------------------------------------------------------------------------------------------
\section{Einleitung}
\sectionauthor{Sergei Pravdin}



\newpage

%--Änderungen-----------------------------------------------------------------------------------------------------------------------------------------------------------------------
\section{Änderungen gegenüber dem Pflichtenheft}
\sectionauthor{Sergei Pravdin}



\newpage

%--Zusatztests-------------------------------------------------------------------------------------------------------------------------------------------------------------------------
\section{Zusätzliche Tests}
\sectionauthor{Jonas Picker}



\newpage

%--Testergebnisse--------------------------------------------------------------------------------------------------------------------------------------------------------------------
\section{Testergebnisse}
\sectionauthor{León Liehr}



\newpage

%--Überdeckungswerte--------------------------------------------------------------------------------------------------------------------------------------------------------------
\section{Überdeckungswerte}
\sectionauthor{Ivan Charviakou}

Im Folgenden werden die Überdeckungswerte des Test-Suites tabellarisch angegeben und danach interpretiert. 
Insbesondere werden diese Werte in zwei Kategorien eingeordnet: Zur Anweisungsüberdeckung gehörend und zur Zweigüberdeckung gehörend. 
Während die Anweisungsüberdeckung die Anzahl an evaluierten Anweisungen misst, beachtet die Zweigüberdeckung die verfolgte Ausführungspfade im Kontrollflussgraphen des Programms.
Somit impliziert eine komplette Zweigüberdeckung eine höhere Anweisungsüberdeckung. 

\subsection{Whitebox-Tests}

\subsection{Blackbox-Tests}

\subsection{Alle Tests}

\newpage

%--Reference---------------------------------------------------------------------------------------------------------------------------------------------------------------------------
\section{Reference}
\sectionauthor{Wario}

\subsection{Nice looking table}
\begin{longtable}{@{}lllll@{}}
\toprule
\textbf{Header 1} & \textbf{Header 2} & \textbf{Header 3} & \textbf{Header 4} & \textbf{Header 5} \\* \midrule
\endfirsthead
%
\endhead
%
Text 1            & Text 2            & Text 3            & Text 4            & Text 5            \\
Text 1            & Text 2            & Text 3            & Text 4            & Text 5            \\* \bottomrule
\end{longtable}

\subsection{Nice looking landscape-table}
\begin{landscape}
\begin{longtable}{@{}lllll@{}}
\toprule
\textbf{Header 1} & \textbf{Header 2} & \textbf{Header 3} & \textbf{Header 4} & \textbf{Header 5} \\* \midrule
\endfirsthead
%
\endhead
%
Text 1            & Text 2            & Text 3            & Text 4            & Text 5            \\
Text 1            & Text 2            & Text 3            & Text 4            & Text 5            \\* \bottomrule
\end{longtable}
\end{landscape}

\newpage

\end{document}


