
% !TeX spellcheck = de_DE

\documentclass{article}

\usepackage[ngerman]{babel}
\usepackage{graphicx}
\usepackage{indentfirst}
\usepackage{hyperref}
\usepackage{geometry}
\usepackage{changepage}
\usepackage{booktabs}
\usepackage{float}
\usepackage{tabulary}
\usepackage{xcolor}
\usepackage{multirow}
\usepackage{caption}
\usepackage{subcaption}
\usepackage{lscape}
\usepackage{colortbl}
\usepackage{listings}

\graphicspath{ {./images/} }
\setlength\parindent{0pt}

\hypersetup{
    colorlinks,
    linkcolor={cyan!50!black},
    citecolor={blue!50!black},
    urlcolor={blue!80!black}
}

\makeatletter
\newcommand{\sectionauthor}[1]{
	{\parindent 0em \large \scshape Autor: #1 \par \nobreak \vspace*{1em}}
	\@afterheading
}
\newcommand{\specification}[3]{
	{\parindent 0.5em \hangindent 3em \hypertarget{spec:#1:#2}{\textbf{/#1#2/}} #3 \par \nobreak \vspace*{0.5em}}
}
\makeatother

\title{Bibliotheksanwendung - Feinspezifikation}
\date{\today\\v1.1}
\author{
	Ivan Charviakou\\
	León Liehr\\
	Mohamad Najjar\\
	Jonas Picker\\
	Sergei Pravdin
}

\begin{document}
\maketitle
\begin{figure}[H]
	\centering
	\includegraphics[width = 30em]{Logo}
\end{figure}
\newpage
\tableofcontents
\newpage

%----------------------------------------------------------------------Kapitel 1--------------------------------------------------------------------------------------------
\section{Einleitung}
In diesem Dokument ist der Implementierungsplan der Webanwendung \textbf{BiBi} dokumentiert. Dabei  erfolgt die Gliederung der Implementierung in Milestones, welche wiederum in Arbeitspakete aufgeteilt werden. Außerdem sind das \textbf{PERT-Diagramm},   die \textbf{Spezialgebiete}-Tabelle und die \textbf{Whitebox-Tests} zu sehen.

\section{Meilensteine}
\sectionauthor{Jonas Picker}
Die Implementierungsphase wird drei Wochen dauern und in 3 gleiche Zeitabschnitte unterteilt, in denen wesentliche Funktionalitäten fertiggestellt werden sollen. Allgemein liegt der Schwerpunkt auf einer vertikale, testbare Realisierung einzelner Funktionen, jedoch müssen gewisse Kernmodule direkt im ersten Milestone abgearbeitet werden.
\subsection{Meilenstein 1}
\textbf{Fertigstellungsdatum:} 04.06.2021 \\
Im Fokus des ersten Meilensteins liegen Grundfunktionalitäten, nach seinem Erreichen soll das System auf normalem Wege gestartet und heruntergefahren werden können. Hierzu werden zunächst die Datenbankinitiallisierung und Anbindung, Fehlerbehandlung sowie System- und E-Mail-Konfiguration als horizontale Funktionspakete erstellt. Geplante vertikale Module des ersten Abschnitts beinhalten den Login und Registriervorgang für Benutzer sowie die Verwaltungsseite für Administratoren.
\subsection{Meilenstein 2}
\textbf{Fertigstellungsdatum:} 11.06.2021 \\
Der Hauptteil der essentiellen Systemfunktionen sollen im zweiten Abschnitt als vertikale Komponenten ausgearbeitet werden. Automatische Abläufe sowie die systemweite Sicherheit und Berechtigungsprüfung stehen an erster Stelle. Nach Erreichen des Meilensteins sollen Nutzer ihre Profildaten (außer E-Mail Reverifizierung) einsehen und verändern können, außerdem soll der Kategoriestöberer fertiggestellt werden. Die Mediumsansicht und Exemplarausleihe sowie Abarbeitungsfunktionalitäten für Mitarbeiter sind ebenfalls zu implementieren. Im gleichen Zug soll dann auch die Medienschemabearbeitung und das Erstellen neuer Medien ermöglicht werden.
\subsection{Meilenstein 3}
\textbf{Fertigstellungsdatum:} 18.06.2021 \\
In der letzten Implementierungsphase soll das System hauptsächlich um Randfunktionen ergänzt werden. Als letzter essentieller Block wird die Suchfunktionalität für Medien fertiggestellt. Mitarbeiter sollen eine Listenansicht mit abzuholenden Exemplaren und Leihfristverstößen angezeigt bekommen können. Für Nutzer soll ebenfalls eine Liste mit ausgeliehenen/abzuholenden Exemplaren einsehbar sein. Auch wird Nutzern die Passwort- und E-Mail-Zurücksetzung ermöglicht. Kleinere Komfortfunktionen wie z.B. das Anzeigen von Suchvorschlägen folgen zuletzt.
%----------------------------------------------------------------------Kapitel 2--------------------------------------------------------------------------------------------

\section{Arbeitspakete}

Die folgende Tabellen dienen der Übersicht über die einzelnen Arbeitspakete, die zur Realisierung unseren System benötigt werden.\\
Als Konvention wird vereinbart, dass wenn nur der Klassenname angegeben wird, die gesamte Klasse
einschließlich aller Methoden zu implementieren ist. Wird zusätzlich einer der Methodenbezeichner angegeben, ist nur
die Methode.



%----------------------------------------------------------------------milestone1 --------------------------------------------------------------------------------------------
\subsection{Meilenstein 1}
Im Folgenden werden die Arbeitspakete, die zum Erreichen des ersten Milestones notwendig sind, aufgelistet und kurz beschrieben. Dabei steht BB für Backing Bean, C für Klasse, CC für Komponente, Co für Konverter, F für Facelet und M für Methode.



\begin{center}

    \begin{table}[H]

      \begin{tabular} {| p{3cm} | p{6cm} |  p{3cm}  | p{2.5cm}|  }
		\hline
	     \textbf{Arbeitspaket}& \textbf{Inhalt} & \textbf{Entwickler} & \textbf{Dauer}\\
	     \hline\hline
	     Meta-implementation & Logger \newline ConfigReader& Ivan Charviakou& ?\\ \hline
	     Utilities I & ConnectionPool \newline ApplicationDao\newline ApplicationDto\newline CSSReader& Jonas Picker& ?\\
	     \hline

	     Utilities II & EmailUtility \newline TokenGenerator\newline HashingModule\newline & Ivan Charviakour& ?\\
	     \hline

	     System I & SystemStartStop \newline DataLayerInitializer\newline Datenbankinitialisierung& Jonas Picker& ?\\
	     \hline

	    Faccelets I & Template \newline Header\newline Footer\newline PaginatedList und column &  León Liehr& ?\\
	    \hline

	      Exception handling & Error  \newline ErrorDto\newline Exceptions\newline Global exception handler &  Sergei Pravdin& ?\\
	     \hline

	       User definition & UserDto  \newline UserSession\newline RoleConverter\newline PasswordValidator\newline ConfirmPasswordValidator \newline EmailValidator \newline UserDao &  Mohamad Najjar& ?\\
	     \hline


	     User authentication & UserDao  \newline Registration\newline Login\newline UserValidator &  Mohamad Najjar& ?\\
	     \hline

	   Application management global & SiteNotice  \newline Contact\newline PrivacyPlocy &  León Liehr & ?\\
	   \hline

	    Administrative functions& Administration  \newline LogoValidator &  Sergei Pravdin& ?\\
	   \hline

      \end{tabular}
      \caption{Tabelle des ersten Milestones}
    \end{table}
\end{center}

\newgeometry{top=1cm,bottom=1cm}
\begin{landscape}
	
	\begin{table}[H]
		\centering
		\begin{tabular}{llllll}
			\toprule
			\textbf{Arbeitspaket} &
			\textbf{ID} &
			\textbf{Inhalt} &
			\textbf{Abhängigkeiten} &
			\textbf{Entwickler} &
			\textbf{Dauer} \\
			\midrule
			Basis &
			11 &
			\begin{tabular}[c]{@{}l@{}}Logger (C)\\ ConfigReader (C)\\ \end{tabular} &
			11 &
			 Ivan Charviakou&
			5 \\
			Utilities I&
			10 &
			\begin{tabular}[c]{@{}l@{}}ConnectionPool  (C)\\ ApplicationDao (C)\\ ApplicationDto (C)\\ CSSReader (C)\\ \end{tabular} &
			11 &
			Jonas Picker &
			4  \\
			Utilities II&
			20 &
			\begin{tabular}[c]{@{}l@{}} EmailUtility (C)\\ TokenGenerator (C)\\ PasswordHashingModule (C)\end{tabular} &
			11 &
			Ivan Charviakou&
			4\\
			Systemstart&
			30 &
			\begin{tabular}[c]{@{}l@{}}SystemStartStop (C)\\DataLayerInitializer (C)\\ DatenbankInitialisierung (SQL)
		\end{tabular} &
			10 &
			Jonas Picker &
			4 \\
			Facelets &
			40 &
			\begin{tabular}[c]{@{}l@{}}Template (F) (BB)\\ PaginatedList (F)\\ \end{tabular} &
			10 &
			León Liehr &
			4 \\
			Fehlerbehandlung &
			41 &
		\begin{tabular}[c]{@{}l@{}}Fehlerseiten  (F) (BB)\\ ErrorDto (FC)\\ ExceptionHandler (C)\\ CustomExceptionHandler (C) \end{tabular}&
			40 &
			Sergei Pravdin &
			3 \\
			Nutzerdefinierung &
			52 &
			\begin{tabular}[c]{@{}l@{}}UserDto  (C) \\ UserSession (MB)\\ RoleConverter (Co)\\ PasswordValidator (C)\\ ConfirmPasswordValidator (C)\\ EmailValidator (C) \\ UserDao (M)\end{tabular}&
			30, 41 &
			Mohamad Najjar &
			6\\
			
			Nutzerathentifizierung &
			50 &
			\begin{tabular}[c]{@{}l@{}}UserDao  (M) \\ Registration (BB) (F)\\ Login (BB) (F)\\ UserValidator (C)\\ ConfirmPasswordValidator (C)\\ EmailValidator (C) \\ UserDao (M)\end{tabular}&
			52, 20 &
			Mohamad Najjar &
			4\\
			
			Globale Anwendungseinstellungen &
			61 &
			\begin{tabular}[c]{@{}l@{}}UserDao  (M) \\ Contact (BB) (F)\\ SiteNotice (BB) (F)\\ PrivacyPlocy (BB) (F)\\\end{tabular}&
			30, 41 &
			León Liehr &
			3\\
			
			Administrative Funktionalitäten &
			60 &
			\begin{tabular}[c]{@{}l@{}}Administration   (BB) (F) \\ LogoValidator (C)\\  \\\end{tabular}&
			30, 41 &
			Sergei Pravdin &
			4\\
			\bottomrule
		\end{tabular}
	 \caption{Tabelle des ersten Milestones}
	\end{table}
	
\end{landscape}

\restoregeometry



%----------------------------------------------------------------------milestone2 --------------------------------------------------------------------------------------------

\subsection{Meilenstein 2}
\newpage
%----------------------------------------------------------------------Milestone3 --------------------------------------------------------------------------------------------

\subsection{Meilenstein 3}
\sectionauthor{León Liehr}

Nachfolgend die Arbeitspakete des dritten Milestones. Dabei steht BB für Backing Bean, C für Klasse, CC für Komponente, Co für Konverter, F für Facelet und M für Methode.

\newgeometry{top=1cm,bottom=1cm}
\begin{landscape}

\begin{table}[H]
    \centering
    \begin{tabular}{llllll}
        \toprule
        \textbf{Arbeitspaket} &
        \textbf{ID} &
        \textbf{Inhalt} &
        \textbf{Abhängigkeiten} &
        \textbf{Entwickler} &
        \textbf{Dauer} \\
        \midrule
        Medium Search &
        130 &
        \begin{tabular}[c]{@{}l@{}}medium-search (F)\\ MediumSearch (BB)\\ searchMedium (M)\\ addNuancedSearchField (M)\\ MediumDao (C)\\ readMediaBySearchCriteria (M)MediumSearchDto (C)\\ AttributeOrCategoryConverter (Co)\\ AvailabilityStatusConverter (Co)\\ MediumPeviewPositionConverter (Co)\\ SearchOperatorConverter (Co)\end{tabular} &
        80, 81 &
        León Liehr &
        8 \\
        Password Reset \& Email Validation &
        150 &
        \begin{tabular}[c]{@{}l@{}}password-reset (F)\\ PasswordReset (BB)\\ resetPassword (M)\\ email-confirmation (F)\\ EmailConfirmation (BB)\\ confirmEmailAddress (M)\\ UserDao (C)\end{tabular} &
        120 &
        Ivan Charviakou &
        6 \\
        Copies of a User &
        100 &
        \begin{tabular}[c]{@{}l@{}}copies-ready-for-pickup (F)\\ CopiesReadyForPickup (BB)\\ borrowed-copies (F)\\ BorrowedCopies (BB)\\ MediumDao (C)\end{tabular} &
        80, 120 &
        Sergei Pravdin &
        5 \\
        Copies by User &
        91 &
        \begin{tabular}[c]{@{}l@{}}copies-ready-for-pickup-all-users (F)\\ CopiesReadyForPickupAllUsers (BB)\\ lending-period-violations (F)\\ LendingPeriodViolations (BB)\\ MediumDao (C)\end{tabular} &
        80, 120 &
        Jonas Picker &
        5 \\
        User Search &
        140 &
        \begin{tabular}[c]{@{}l@{}}user-search (F)\\ UserSearch (BB)\\ UserSearchDto (C)\\ UserDao (C)\end{tabular} &
        120 &
        Mohamad Najjar &
        6 \\
        Enhancements &
        160 &
        reactiveInputField (CC) &
        130, 140, 110 &
        León Liehr &
        3 \\
        \bottomrule
    \end{tabular}
    \caption{Tabelle des drtten Milestones}
\end{table}

\end{landscape}
\restoregeometry

\section{PERT-Diagramm}

%----------------------------------------------------------------------Kapitel 4--------------------------------------------------------------------------------------------

\section{Spezialgebiete}
\sectionauthor{Jonas Picker}
Die Vielzahl der verwendeten Technologien erfordert eine Spezialisierung der einzelnen Teammitglieder. Die jeweiligen Spezialgebiete sind \hyperlink{speziell}{unten} aufgelistet.
\begin{table}[H]
\centering
\hypertarget{speziell}{}
\begin{tabular}{| p{6cm} | p{6cm} |}
	\hline
     	git & León Liehr \\
     	\hline
     	JSF Internationalisierung & León Liehr \\
     	\hline
    	JSF Templates & León Liehr \\
     	\hline
     	JSF Components & León Liehr \\
     	\hline
     	\hline
     	LaTeX & Ivan Charviakou \\
     	\hline
     	RegEx & Ivan Charviakou \\
     	\hline
     	Jakarta Mail & Ivan Charviakou \\
     	\hline
     	JSF Validators & Ivan Charviakou \\
     	\hline
     	\hline
     	JSF Converters & Mohamad Najjar \\
    	\hline
    	 RSA & Mohamad Najjar \\
    	\hline
    	 CSS/Bootstrap & Mohamad Najjar \\
     	\hline
     	Listenabstraktion & Mohamad Najjar \\
     	\hline
     	\hline
     	SQL & Jonas Picker \\
    	\hline
    	JSF File Upload & Jonas Picker \\
     	\hline
     	JSF/CDI Scopes & Jonas Picker \\
     	\hline
     	SSL/TLS & Jonas Picker \\
     	\hline
     	Systemkonfiguration & Jonas Picker \\
     	\hline
     	\hline
     	Selenium & Sergei Pravdin \\
     	\hline
     	Logging & Sergei Pravdin \\
     	\hline
     	JUnit & Sergei Pravdin \\
     	\hline
\end{tabular}
\end{table}

%----------------------------------------------------------------------Kapitel 5--------------------------------------------------------------------------------------------
\section{Whitebox-Tests}

\end{document}
