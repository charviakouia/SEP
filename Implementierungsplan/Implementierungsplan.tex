
% !TeX spellcheck = de_DE

\documentclass{article}

\usepackage[ngerman]{babel}
\usepackage{graphicx}
\usepackage{indentfirst}
\usepackage{hyperref}
\usepackage{geometry}
\usepackage{changepage}
\usepackage{booktabs}
\usepackage{float}
\usepackage{tabulary}
\usepackage{xcolor}
\usepackage{multirow}
\usepackage{caption}
\usepackage{subcaption}
\usepackage{lscape}
\usepackage{colortbl}
\usepackage{listings}

\graphicspath{ {./images/} }
\setlength\parindent{0pt}

\hypersetup{
    colorlinks,
    linkcolor={cyan!50!black},
    citecolor={blue!50!black},
    urlcolor={blue!80!black}
}

\makeatletter
\newcommand{\sectionauthor}[1]{
	{\parindent 0em \large \scshape Autor: #1 \par \nobreak \vspace*{1em}}
	\@afterheading
}
\newcommand{\specification}[3]{
	{\parindent 0.5em \hangindent 3em \hypertarget{spec:#1:#2}{\textbf{/#1#2/}} #3 \par \nobreak \vspace*{0.5em}}
}
\makeatother

\title{Bibliotheksanwendung - Feinspezifikation}
\date{\today\\v1.1}
\author{
	Ivan Charviakou\\
	León Liehr\\
	Mohamad Najjar\\
	Jonas Picker\\
	Sergei Pravdin
}

\begin{document}
\maketitle
\begin{figure}[H]
	\centering
	\includegraphics[width = 30em]{Logo}
\end{figure}
\newpage
\tableofcontents
\newpage

%----------------------------------------------------------------------Kapitel 1--------------------------------------------------------------------------------------------

\section{Meilensteine}
\sectionauthor{Jonas Picker}
Die Implementierungsphase wird drei Wochen dauern und in 3 gleiche Zeitabschnitte unterteilt, in denen wesentliche Funktionalitäten fertiggestellt werden sollen. Allgemein liegt der Schwerpunkt auf einer vertikalen, testbaren Realisierung einzelner Funktionen, jedoch müssen gewisse Kernmodule direkt im ersten Milestone abgearbeitet werden. 
\subsection{Meilenstein 1}
\textbf{Fertigstellungsdatum:} 04.06.2021 \\
Im Fokus des ersten Meilensteins liegen Grundfunktionalitäten, nach seinem Erreichen soll das System auf normalem Wege gestartet und heruntergefahren werden können. Hierzu werden zunächst die Datenbankinitiallisierung (mit Beispieldaten) und Anbindung sowie der Logger und die Systemkonfiguration als horizontale Funktionspakete erstellt. Geplante vertikale Module des ersten Abschnitts beinhalten den Login, eine paginierte Listenansicht und die auf allen Facelets sichtbaren Seiten- und Kopfleisten. 
\subsection{Meilenstein 2}
\textbf{Fertigstellungsdatum:} 11.06.2021 \\
Der Hauptteil der essentiellen Systemfunktionen sollen im zweiten Abschnitt als vertikale Komponenten ausgearbeitet werden. Die Mediumsansicht/erstellung und Suchfunktionalitäten für Medien stehen dabei im Vordergrund. Nach Erreichen des Milestones soll nach Kategorien gestöbert und neue Kategorien sollen erstellt werden können. Neben dem Nutzerprofil und der Registrierung werden außerdem Listenansichten für Mitarbeiter fertiggestellt werden.
\subsection{Meilenstein 3}
\textbf{Fertigstellungsdatum:} 18.06.2021 \\
In der letzten Implementierungsphase soll das System hauptsächlich um Randfunktionen ergänzt werden. Als letzte essentieller Blöcke werden die Anwendungseinstellungen für Admins, die Suchfunktionalität für Nutzer und der Wartungsthread fertiggestellt. Außerdem wird die Zugriffskontrolle mittels einer PhaseListener-Implementierung realisiert. Mitarbeiter sollen eine Listenansicht mit  Leihfristverstößen angezeigt bekommen können. Für Nutzer soll ebenfalls eine Liste mit ausgeliehenen/abzuholenden Exemplaren einsehbar sein. Kleinere Komfortfunktionen wie z.B. das Anzeigen von Suchvorschlägen folgen zuletzt.
%----------------------------------------------------------------------Kapitel 2--------------------------------------------------------------------------------------------

\section{Arbeitspakete}

%----------------------------------------------------------------------Kapitel 3--------------------------------------------------------------------------------------------

\section{PERT-Diagramm}

%----------------------------------------------------------------------Kapitel 4--------------------------------------------------------------------------------------------

\section{Spezialgebiete}
\sectionauthor{Jonas Picker}
Die Vielzahl der verwendeten Technologien erfordert eine Spezialisierung der einzelnen Teammitglieder. Die jeweiligen Spezialgebiete sind \hyperlink{speziell}{unten} aufgelistet.
\begin{table}[H]
\centering
\hypertarget{speziell}{}
\begin{tabular}{| p{6cm} | p{6cm} |}
	\hline
     	git & León Liehr \\
     	\hline
     	JSF Internationalisierung & León Liehr \\
     	\hline
    	JSF Templates & León Liehr \\	
     	\hline
     	JSF Components & León Liehr \\
     	\hline
     	Listenabstraktion & León Liehr \\
     	\hline
     	Suchfunktionalitäten & León Liehr \\
     	\hline
     	\hline
     	LaTeX & Ivan Charviakou \\
     	\hline
     	RegEx & Ivan Charviakou \\
     	\hline
     	Jakarta Mail & Ivan Charviakou \\
     	\hline
     	JSF Validators & Ivan Charviakou \\
     	\hline
     	Design-Patterns & Ivan Charviakou \\
     	\hline
     	\hline
     	JSF Converters & Mohamad Najjar \\
    	\hline
    	 RSA & Mohamad Najjar \\
    	\hline
    	 CSS/Bootstrap & Mohamad Najjar \\
     	\hline
     	Nutzerauthentifizierung & Mohamad Najjar \\
     	\hline
     	\hline
     	SQL & Jonas Picker \\
    	\hline
    	Logging & Jonas Picker \\
     	\hline
     	JSF/CDI Scopes & Jonas Picker \\
     	\hline
     	SSL/TLS & Jonas Picker \\
     	\hline
     	Systemkonfiguration & Jonas Picker \\
     	\hline
     	\hline
     	Selenium & Sergei Pravdin \\
     	\hline
     	JSF File Upload & Sergei Pravdin \\
     	\hline
     	JUnit & Sergei Pravdin \\
     	\hline
     	
\end{tabular}
\end{table}

%----------------------------------------------------------------------Kapitel 5--------------------------------------------------------------------------------------------
\section{Whitebox-Tests}

\end{document}
