
\documentclass{article}

\usepackage{graphicx}
\usepackage{indentfirst}
\usepackage{hyperref}

\graphicspath{ {./images/} }

\makeatletter
\newcommand{\sectionauthor}[1]{
	{\parindent 0em \large \scshape Autor: #1 \par \nobreak \vspace*{2em}}
	\@afterheading
}
\newcommand{\specification}[3]{
	{\parindent 0.5em \hangindent 3em \hypertarget{spec:#1:#2}{\textbf{/#1#2/}} #3 \par \nobreak \vspace*{0.5em}}
}
\makeatother

\title{Bibliothekanwendung - Pflichtenheft}
\date{\today}
\author{
	Ivan Charviakou\\
	León Liehr\\
	Mohamad Najjar\\
	Jonas Picker\\
	Sergei Pravdin
}

\begin{document}

\maketitle
\begin{figure}[h]
	\centering
	\includegraphics[width = 20em]{Dedede}
\end{figure}
\newpage

\section{Einleitung} %-------------------------------------------------------------------------------------------------
\sectionauthor{Jonas Picker}
Aufbauend auf dem Lastenheft von Christian Bachmaier und Armin Größlinger für ein Bibliothekssystem steckt unser Team, bestehend aus Sergei Pravdin, Ivan Charviakou, León Liehr, Mohamad Najjar und Jonas Picker, mit diesem Pflichtenheft den Rahmen der zu erbringenden Leistungen und verwendeten Technologien bei der Bearbeitung des Auftrags ab. Bei dem zu erstellenden System handelt es sich um eine vereinfachte Form eines Bibliotheksverwaltungssystems mit dem der Anwender das mediale Bibliotheksinventar zentral durchsuchen, kategorisieren und die Nutzung verwalten kann. Das fertige Produkt wird über einen Webbrowser bedient und bietet zahlreiche Anpassungsmöglichkeiten. Das restliche Dokument beinhaltet die genauen Spezifikationen der zu implementierenden Funktionalitäten.

\section{Zielbestimmung} %-------------------------------------------------------------------------------------------------
\sectionauthor{Jonas Picker}

\subsection{Musskriterien}
Bei der Produktinstallation auf dem Server konfiguriert der Betreiber den Webspace, spezifiziert den anzusteuernden E-Mail-Server und die zur Speicherung verwendete Datenbank. Im laufenden System zählt der Betreiber zu den Administratoren. Technische Maßnahmen sorgen dafür, dass jegliche sensible Information sicher übertragen sowie gespeichert wird und kein unbefugter Zugriff auf zugangsbeschränkte Bereiche der Anwendung durch Dritte erfolgt. Eine unterstützende Bedienanleitung steht für jeden nicht offensichtlichen Aspekt der Anwendung online zur Verfügung.

\begin{flushleft}
\textbf{Administratoren:} Der Betreiber ist zunächst der einzige Administrator. Administratoren können authentifizierte Nutzer in den Administratorstatus erheben sowie andere Administratoren degradieren. Administratoren sind für die Anwendungskonfiguration zuständig, welche das Setzen der institutionsspezifischen Eigenschaften (Logos, Namen, Impressum, Datenschutzerklärung etc.) sowie Variationen des Look \& Feels beinhalten. Außerdem bestimmen sie Registrierungsbedingungen und Zugangsberechtigungen für anonyme Nutzer. Ein Administrator kann die Menge der Attribute, welche alle Medien aufweisen, global bestimmen. Auch kann er die Attributwerte der einzelnen Medieninstanzen editieren, sowie Medieninstanzen erstellen, löschen und sie in hierarchische Kategorien (bis Tiefe 2) einteilen. Verwaltungs- und Übersichtsfunktionen zur Ausleihe und Rückgabe von Medien und den damit verbundenen Fristen werden außerdem bereitgestellt. Administratoren können andere Nutzerkonten editieren und löschen, hierfür steht eine Suchfunktion bereit. Die Funktionen der authentifizierten Benutzer (und Bibliotheksmitarbeiter) stehen auch den Administratoren zur Verfügung.
\end{flushleft}

\begin{flushleft}
\textbf{Authentifizierte Nutzer:} Nach einer Validierung der E-Mail-Adresse ist ein Nutzer im System registriert und kann sich jederzeit mit seinen Accountdaten anmelden, danach kann diese verändern und er kann sich verfügbare Exemplare von Medien zur Abholung markieren, hierzu stehen ihm die im nächsten Paragraph beschriebenen Such- und Browsingfunktionen zur Verfügung. Wird das Exemplar innerhalb einer gesetzten Frist abgeholt, gilt es als ausgeliehen und der authentifizierte Nutzer wird per E-Mail automatisch über den Ablauf der Rückgabefrist informiert. Ein angemeldeter Benutzer kann seinen Account selbstständig löschen (außer er ist der letzte Administrator) und sich jederzeit abmelden.
\end{flushleft}

\begin{flushleft}
\textbf{Anonyme Nutzer:} In der Standartkonfiguration erlaubt das System Besuchern des Webspaces, neben der Registrierungsmöglichkeit, den Zugriff auf eine Such- und Browsingfunktion sowie das Herunterladen/Lesen öffentlich zugänglicher Medien. Diese Berechtigungen können von Administratoren auf die Registrierungsmöglichkeit reduziert werden. Die Suchfunktion beinhaltet die Möglichkeit sowohl nach Medien als auch nach deren Attributwerten zu suchen, sowie nach Kategorien zu filtern. Eine Detailansicht zu gefundenen Medien steht ebenfalls zur Verfügung. Um sich einen Überblick über die angebotenen Medien zu verschaffen, lässt sich die Kategoriehierarchie in Baumform anzeigen.
\end{flushleft}

\subsection{Wunschkriterien}

\begin{flushleft}
\textbf{Administratoren:} Beim der Zuweisung der Kategorien für Medien ist die Hierarchietiefe unbeschränkt. Zusätzlich besitzen Administratoren die Möglichkeit, einzelnen registrierten Nutzer ohne Administratorberechtigungen die Ausleihberechtigung zu entziehen bzw. zu verleihen. Hierzu gibt es eine gesammelte Ansicht der noch nicht freigeschalteten Accounts.
\end{flushleft}

\begin{flushleft}
\textbf{Authentifizierte Nutzer:} Es besteht die Möglichkeit die mit diesem Account ausgeliehenen Medienexemplare einzusehen.
\end{flushleft}

\begin{flushleft}
\textbf{Bibliotheksmitarbeiter:} Diese Nutzerrolle kann Ausleihen (direkt am Schalter und online angekündigte) abarbeiten. Um eine Abarbeitung der zur Abholung markierten Medien zu erleichtern steht eine Listenansicht zur Verfügung. Außerdem können sie zurückgegebene Ausleihen wieder als verfügbar markieren. Zusätzlich ist es ihnen möglich, Attributwerte für Medien und Exemplare zu editieren und neue Medien/Exemplare zu erstellen.
\end{flushleft}

\subsection{Abgrenzungskriterien}

Der Hauptfokus des Produkts liegt auf der Organisation und Verwaltung von Medien und Nutzern der Bibliothek, Funktionalitäten zum Erwerben und Verwalten von Lizenzen und/oder neuen Medien stehen nicht zur Verfügung. Die Abwicklung oder Verfolgung der Kundenzahlungen für Mitgliedsbeiträge, Medienerwerb o.ä. ist ebenfalls nicht im System integriert. Es besteht keine Möglichkeit, andere Bibliotheksmanagementsysteme oder Literaturenzyklopädien mit dieser Software zu verknüpfen. Das System ist für die klassische Benutzung über einen mit Tastatur, Maus und Bildschirm ausgestatteten PC/Laptop gedacht, mobile Geräte oder barrierefreie Benutzung werden begrenzt bis gar nicht unterstützt. Die Anwendung richtet sich an eine deutsche Zielgruppe und wird in der deutschen Sprache ausgeliefert. Obwohl all diese Kriterien initial nicht enthalten sind, erlaubt das modulare Entwicklungsprinzip eine prinzipielle Erweiterbarkeit der Anwendung um Zusatzfunktionen.

\newpage

\section{Produkteinsatz} %-------------------------------------------------------------------------------------------------

\newpage

\section{Produktumgebung} %-------------------------------------------------------------------------------------------------
\sectionauthor{Jonas Picker}

\subsection{Software}

Im folgenden Abschnitt werden Softwareabhängigkeiten und Kompatibilitäten des Produkts beschrieben.

\begin{itemize}
\item \underline{\textbf{Clientsoftware}}: \linebreak
Die Applikation wird über einen Webbrowser benutzt, dieser sollte die Darstellung in den Auszeichnungssprachen HTML, Version 5 und CSS, Level 3 unterstützen, sowie die Kommunikation mit dem HTTP/1.1-Protokoll. Die Unterstützung von JavaScript (ECMAScript 2020) im Browser ist ebenfalls vorausgesetzt. Explizit getestet werden die aktuellen Versionen der weit verbreiteten Browser: Google Chrome, Version: 88.0; Mozilla Firefox, Version: 85.0 und Apple Safari, Version: 14.0.3. Da HTML 5 (und CSS) jedoch seit einem guten Jahrzehnt als Web-Standart gilt, ist weitreichende Abwärtskompatibilität bei den meisten Browsern zu erwarten.  

\item \underline{\textbf{Serversoftware}}: \linebreak
Auf dem Server muss die Laufzeitumgebung 'Java Virtual Maschine' verfügbar sein, hierzu muss das 'Development Kit' OpenJDK 15.0.2 installiert und beim ebenfalls benötigten Java Enterprise Applikationsserver Apache Tomcat, Version 10.0.x registriert werden. Außerdem muss ein E-Mail-Server bereit stehen, der das SMTP-Protokoll unterstützt. Die verwendete Datenbank muss das objektrelationale Datenbankmanagementsystem PostgreSQL, Version 12 benutzen. Die Installation der JVM/Tomcat, das Aufspielen der Anwendung und des Zertifikats (siehe Orgware) sowie die Registrierung der Datenbank und des E-Mail-Servers werden in einem Installationsdokument beschrieben.
\end{itemize}

\subsection{Hardware}

Hier werden die geschätzen Rahmenbedingungen der Hardware für einen reibungslosen Betrieb aufgelistet. Sollten die Rahmenbedingungen unterschritten werden, ist ein Betrieb meistens immer noch möglich, jedoch nicht garantiert.

\begin{itemize}
\item \underline{\textbf{Clienthardware}}: \linebreak
Die Bedienung der Anwendung über die oben beschriebenen Browsertypen setzt voraus, dass Clientrechner die jeweiligen Hardwareanforderungen für deren Betrieb erfüllen.
\item \underline{\textbf{Serverhardware}}: \linebreak
Die Mindestanforderungen zum Betreiben einer Datenbank unter PostgreSQL bzw. eines Applikationsservers unter Tomcat reichen für einen sinnvollen Betrieb der Anwendung unter Last nicht aus, die der Anwendung zur Verfügung stehenden Ressourcen sollten mit der Nutzerzahl in Relation stehen. Zur Speicherung der Nutzerprofile/Medien muss je nach Umfang ausreichend Datenbankspeicherplatz zur Verfügung stehen, für den anfänglichen Betrieb mindestens 1TB. Mehr Arbeitsspeicher, sowie höhere Bandbreite der Datenbank- und Internetanbindung ermöglicht dem Server ein effizienteres Stemmen vieler gleichzeitiger Nutzer. Die Architektur des Systems erlaubt es der Anwendung mit den zur Verfügung stehenden Ressourcen zu skalieren. Als Referenzplattform zum Betreiben der Anwendung dient der unten beschriebene Rechner 'schratz' für Datenbank und Server gleichermaßen.
\item \underline{\textbf{Referenzrechner}}:
\begin{flushleft}
CIP-Pool Computer 'schratz', Uni Passau, Betriebssystem: GNU/Linux Debian, Version: 4.19.132-1\\ 15.6GB RAM, Intel Core i7-4790 3.60GHz CPU, Intel I217-LM Ethernet Controller \linebreak
Spezifikation der Datenbank bezieht sich auf die Referenzhardware von 'schratz'. Die Übertragungsrate zwischen Server und Datenbank beträgt ca. 1 GBit/s. Die Datenbank unterstützt das Verschlüsselungsprotokoll TLS mit einem Zertifikat. \linebreak
\end{flushleft}
\end{itemize}

\subsection{Orgware und Schnittstellen}

\begin{itemize}
\item \underline{\textbf{Client}}: \linebreak
Die notwendigen Mensch-Maschine Schnittstellen zum Benutzen der Anwendung über einen Browser sind: Bildschirm, Tastatur. Die Benutzung einer Maus oder vergleichbare Cursorsteuerung wird dringend empfohlen. Die Verifizierung eines neuen Benutzers bei der Registrierung im System wird über eine gültige E-Mail-Adresse abgewickelt, folglich muss der Nutzer Zugriff auf ein E-Mail-Konto besitzen. Der Browser des Klienten muss beim Benutzen über eine konstante Anbindung an das Internet (mind. 5 MBit/s) verfügen, eine außergewöhnlich hohe Bandbreite ist nicht erforderlich.
\item \underline{\textbf{Server}}: \linebreak
Der Server muss über notwendige Mensch-Maschine Schnittstellen zur initialen Installation der Anwendung und ihrer Softwareabhängigkeiten verfügen. Da Administratorkonten eine E-Mail-Adresse benötigen, ist bei der Installation der Zugang zu einem E-Mail-Konto erforderlich. Auch benötigt der Server eine konstante Anbindung an das (Intra- und) Internet, um Klientenanfragen beantworten, den E-Mail-Server kontaktieren und Datenbankanfragen absetzen zu können. Die Benötigte Bandbreite ist stark abhängig von der zu erwartenden Zahl der gleichzeitigen Zugriffe, für einen Minimalbetrieb mind. 25MBit/s. Um den Webspace über das Internet aufrufen zu können, muss die Domain einen Eintrag im DNS mit der festen IP des Servers besitzen. Die Datenbank kann sich auf dem gleichen Gerät wie der Server, seperat davon im Netzwerk oder geographisch getrennt 'im Internet'  befinden. Sie muss jedoch vom Server aus über einen Hostnamen ansteuerbar sein, je nach Realisierung ist dazu ein VPN-Tunnel notwendig. Um sichere Datenübertragungen mit dem TLS-Protokoll zu ermöglichen, benötigt der Tomcat-Server ein SSL-Zertifikat. Wenn sich der Server mit der Datenbank über das Internet (via SSL-VPN) verbinden soll, muss auch die Datenbank mit einem SSL-Zertifikat ausgestattet sein und die Verbindung zwischen ihr und dem Server sollte eine Bandbreite von mind. 1GBit/s aufweisen. 
\end{itemize}

\newpage

\section{Produktfunktionen} %-------------------------------------------------------------------------------------------------

\newpage

\section{Produktleistung} %-------------------------------------------------------------------------------------------------

\newpage

\section{Benutzeroberfläche} %-------------------------------------------------------------------------------------------------

\newpage

\section{Qualitätsanforderungen} %-------------------------------------------------------------------------------------------------

\newpage

\section{Testing} %-------------------------------------------------------------------------------------------------

\newpage

\section{Entwicklungsumgebung} %-------------------------------------------------------------------------------------------------
\sectionauthor{Jonas Picker}

Die Entwicklung finden auf verschiedenen Privatrechnern der Teammitglieder statt, deren Bauteile und Betriebssysteme in den folgenden Listen kurz umrissen werden, keiner der Rechner weißt sonstige Besonderheiten auf. 

\subsection{Software}

\begin{itemize}
\item \underline{\textbf{Dokumentenbearbeitung}}: 
\begin{flushleft}
LaTeX Distribution MiKTeX, Version: 21.2 \\
LaTeX-Editor TeXworks, Version: 0.6.6 \linebreak
Overleaf Online Latex Editor \linebreak
LaTeX Distribution MacTeX-2021 \\
LaTeX-Editor TeXShop, Version: 4.62 (siehe IDEs) \linebreak
LaTeX Distribution TeX Live, Version: 2021 \\
Dokumentenbetrachter Evince, Version: 3.38.1 \\
LaTeX-Bearbeitung mit IntelliJ (siehe IDEs) \\
LaTeX-Editor TeXstudio, Version: 3.1.1 \\
\end{flushleft}
\item \underline{\textbf{Webbrowser}}:
\begin{flushleft}
Google Chrome, Version: 88.0 \linebreak
Mozilla Firefox, Version: 85.0 \linebreak
Apple Safari, Version: 14.0.3 \linebreak
\end{flushleft}
\item \underline{\textbf{Integrierte Entwicklungsumgebungen}}: 
\begin{flushleft}
Eclipse IDE for Enterprise Java Developers, Version: 2020-12 (4.18.0) \linebreak
IntelliJ IDEA 2021.1 Ultimate Edition \linebreak
\end{flushleft}
\item \underline{\textbf{Objektorientierte Modellierung}}: 
\begin{flushleft}
IBM Rational Software Architect Designer 9.7 \linebreak
\end{flushleft}
\item \underline{\textbf{Programmiersprachen, Entwicklungs- \& Testframeworks}}: 
\begin{flushleft}
Java OpenJDK JDK 15.0.2 GA-Release von https://jdk.java.net/15/\linebreak
Jakarta EE 9 \linebreak
Jakarta Server Faces 3.0 Mojarra Implementation \linebreak
CDI 3.0 with Red Hat Weld, Version: 4.0.1.Final \linebreak
Cascading Style Sheets Level 3 \linebreak
JUnit, Version: 5.7.1 \linebreak
Selenium Server (Grid), Version: 3.141.59 \linebreak
EclEmma, Version: 3.1.4 \\
Mockito, Version: 3.9.3 \\
\end{flushleft}
\item \underline{\textbf{Versionsmanagement}}:
\begin{flushleft}
git, Version 2.31.1 \linebreak
\end{flushleft}
\item \underline{\textbf{Betriebssysteme}}:
\begin{flushleft}
MacOS Big Sur, Versionen: 11.2.3 und 11.2.1 \linebreak
Windows 10 Home 20H2 \linebreak
GNU/Linux Arch, Version: 5.11.11 \linebreak
GNU/Linux Debian, Version: 4.19.132-1 \linebreak
\end{flushleft}
\item \underline{\textbf{Prototypen/Graphen/Zeichnungen}}:
\begin{flushleft}
Balsamiq Wireframes, Version: 4.2.4 \linebreak 
yEd Graph Editor, Version: 3.21.1 \linebreak 
InkScape, Version: 1.0.2 \\
Dia, Version: 0.97.2 \\
GIMP, Version: 2.10.24 \\
\end{flushleft}
\item \underline{\textbf{Applikationsserver}}: 
\begin{flushleft}
Apache Tomcat, Version: 10.0.2 \linebreak
\end{flushleft}
\item \underline{\textbf{Remote Access}}
\begin{flushleft}
X2Go, Version: 4.1.0.3 \\
OpenVPN, Version: 2.5.1 \\
OpenSSH, Versionen: OpenSSH\_for\_Windows\_7.7p1 und 8.5p1-1\\
\end{flushleft}
\item \underline{\textbf{Datenbanktreiber und Visualisierung}}: 
\begin{flushleft}
PostgreSQL JDBC 4.2 Driver, Version: 42.2.19 \linebreak
DBeaver, Version: 21.0.2 \\
\end{flushleft}
\item \underline{\textbf{Teamkommunikation}}: 
\begin{flushleft}
Slack \linebreak
Discord \linebreak
Skype \linebreak
Stud.IP \linebreak
\end{flushleft}
\end{itemize}

\subsection{Hardware}

\begin{itemize}
\item \underline{\textbf{Entwicklungsrechner}}: 
\begin{flushleft}
MacBook Pro, 8GB RAM, Intel Core i5 2GHz  \linebreak
HP Laptop 15-db1xxx, 16GB RAM, AMD Ryzen 5 3500U 2.1GHz \linebreak
2x MacBook Pro, 8GB RAM, Intel Core i5 2.3GHz \linebreak
Tower-PC, 16GB RAM, AMD Ryzen 5 3600XT 3.8GHz \linebreak
\end{flushleft}
\item \underline{\textbf{Referenzrechner}}:
\begin{flushleft}
Der im Abschnitt 'Produktumgebung' beschriebene CIP-Pool Rechner 'schratz'.
\end{flushleft}
\end{itemize}

\subsection{Entwicklungsschnittstellen}

\begin{itemize}
\item \begin{flushleft} Netzwerk- und Internetverbindung \end{flushleft} 
\item \begin{flushleft} Git-verwaltetes Repository der Uni Passau, Website: https://git.fim.uni-passau.de/ \end{flushleft} 
\item \begin{flushleft} Virtualisierte Datenbank der Uni Passau, Hostname: bueno.fim.uni-passau.de \end{flushleft}
\end{itemize}

\newpage

\section{Glossar} %-------------------------------------------------------------------------------------------------
\sectionauthor{Jonas Picker}
\begin{itemize}\item OPAC (5.1)
\begin{flushleft}
'Online Public Access Catalog', darunter versteht man den über das Internet zugänglichen Bibliothekskatalog, mit dem der Bestand an Publikationen recherchiert werden kann.
\end{flushleft}
\item Webbrowser (Einleitung)
\begin{flushleft}
Spezielle Computerprogramme zum Anzeigen von Webseiten im World Wide Web. Bekannte Beispiele sind: Google Chrome, Apple Safare, Mozilla Firefox.
\end{flushleft}
\item Webspace (2.1)
\begin{flushleft}
Unter einem Webspace versteht man einen Speicherplatz für Dateien auf einem Server, der über das Internet erreichbar ist.
\end{flushleft}
\item Look \& Feel (2.1)
\begin{flushleft}
Die Umschreibung der Kombination aus Farbauswahl und Form sowie Größe und Position der Elemente einer Webseite.
\end{flushleft}
\item HTML (4.1)
\begin{flushleft}
Die 'Hypertext Markup Language' ist eine textbasierte Auszeichnungssprache, die als Dokumentenstandart eine Grundlage des World Wide Webs bildet.
\end{flushleft}
\item CSS (4.1)
\begin{flushleft}
'Cascading Style Sheets' ist eine weitere Auszeichnungssprache mit der unter anderem die Darstellung, Form, Farben und Positionen der Elemente einer Webseite verändert werden kann.
\end{flushleft}
\item JavaScript (4.1)
\begin{flushleft}
Ist eine Skriptsprache, die hauptsächlich dazu verwendet wird, die Logik hinter interaktiven Elementen einer Webseite zu formulieren.
\end{flushleft}
\item Laufzeitumgebung (4.1)
\begin{flushleft}
Hierunter versteht man die für eine Programmiersprache zur Ausführungszeit verfügbaren und festgelegten Voraussetzungen.
\end{flushleft}
\item SSL/TLS (4.1)
\begin{flushleft}
Das 'Secure Socket Layer'- bzw. 'Transport Layer Security'-Protokoll bietet die Möglichkeit zur sicheren Datenübertragung über das Internet, hierzu ist ein kryptographisches Schlüsselzertifikat nötig.
\end{flushleft}
\item SMTP (4.1)
\begin{flushleft}
Das 'Simple Mail Transfer Protocol' wird zum Weiterleiten und Verschicken von E-Mails verwendet.
\end{flushleft}
\item (SSL-)VPN (4.2)
\begin{flushleft}
'Virtual Private Network', der Begriff meint hier den verschlüsselten Fernzugriff auf Ressourcen in einem fremden lokalen Netzwerk über einen 'Tunnel' durch das Internet mit dem TLS-Protokoll.
\end{flushleft}
\item DNS (4.2)
\begin{flushleft}
Das 'Domain Name System' ist für die Namensauflösung der Domainnamen in IP-Adressen im Internet zuständig. Um bspsw. über die Adresszeile im Browser gefunden zu werden, muss ihr Webspace über einen Zwischenhändler (z.B. GoDaddy) im System registriert werden.
\end{flushleft}
\item ID (6.)
\begin{flushleft}
Dies ist die Abkürzung für einen Identifikator, normalerweiße eine für den zu identifizierenden einzigartige Nummer ohne weiteren Informationsgehalt.
\end{flushleft}
\item UTF-8 (7.1)
\begin{flushleft}
Das 8-bit 'Unicode Transformation Format' wird zur Kodierung von nahezu allen weltweit verwendeten Zahlen, Schriftzeichen und sonstigen schriftlichen Elementen verwendet.
\end{flushleft}
\item Paginierung (7.1)
\begin{flushleft}
Hierunter versteht man die Aufteilung von abgefrageten Listen in kleinere Abschnitte, um die Darstellung zu erleichtern und lange Ladezeiten bei großen Ergebnissen zu vermeiden.
\end{flushleft}
\end{itemize}
\end{document}
