
\documentclass{article}

\usepackage{graphicx}
\usepackage{indentfirst}

\graphicspath{ {./images/} }

\makeatletter
\newcommand{\sectionauthor}[1]{
	{\parindent 0em \large \scshape Autor: #1 \par \nobreak \vspace*{2em}}
	\@afterheading
}
\makeatother

\title{Bibliothekanwendung - Pflichtenheft}
\date{\today}
\author{
	Ivan Charviakou\\
	León Liehr\\
	Mohamad Najjar\\
	Jonas Picker\\
	Sergei Pravdin
}

\begin{document}

\maketitle
\begin{figure}[h]
	\centering
	\includegraphics[width = 20em]{Dedede}
\end{figure}
\newpage

\section{Einleitung}
\sectionauthor{Jonas Picker}
Aufbauend auf dem Lastenheft von Christian Bachmaier und Armin Größlinger für ein Bibliothekssystem steckt unser Team, bestehend aus Sergei Pravdin, Ivan Charviakou, León Liehr, Mohamad Najjar und Jonas Picker, mit diesem Pflichtenheft den Rahmen der zu erbringenden Leistungen und verwendeten Technologien bei der Bearbeitung des Auftrags ab. Bei dem zu erstellenden System handelt es sich um eine vereinfachte Form eines Bibliotheksverwaltungssystems mit dem der Anwender das mediale Bibliotheksinventar zentral durchsuchen, kategorisieren und die Nutzung verwalten kann. Das fertige Produkt wird über einen Webbrowser bedient und bietet zahlreiche Anpassungsmöglichkeiten. Das restliche Dokument beinhaltet die genauen Spezifikationen der zu implementierenden Funktionalitäten.

\section{Zielbestimmung}
\sectionauthor{Jonas Picker}

\subsection{Musskriterien}
Bei der Produktinstallation setzt der Betreiber den gewünschten Webspace sowie die vom System verwendete Datenbank und den E-Mail-Server, im laufenden System zählt der Betreiber zu den Administratoren. Technische Maßnahmen sorgen dafür, dass jegliche sensible Information sicher übertragen sowie gespeichert wird und kein unbefugter Zugriff auf zugangsbeschränkte Bereiche der Anwendung durch Dritte erfolgt. Eine unterstützende Bedienanleitung steht für jeden nicht offensichtlichen Aspekt der Anwendung online zur Verfügung.

\begin{flushleft}
\textbf{Administratoren:} In diesen Status werden authentifizierte Benutzer initial vom Betreiber und danach von anderen Administratoren erhoben, dies kann auch rückgängig gemacht werden. Administratoren sind für die Anwendungskonfiguration zuständig, welche das Setzen der institutionsspeezifischen Eigenschaften (Logos, Namen, Impressum, Datenschutzerklärung etc.) sowie Variationen des Look \& Feels beinhalten, außerdem bestimmen sie Registrierungsbedingungen und Zugangsberechtigungen für anonyme Nutzer.  Die Definition von Eigenschaften der vom System verwalteten Medien sowie Festlegen ihrer benutzerdefinierten Kategorien, die hierarchisch (bis Tiefe 2) in Verbindung stehen könnnen, wird den Administratoren zusätzlich zur Bearbeitung der Attributwerte einzelner Medieninstanzen ermöglicht. Verwaltungs- und Übersichtsfunktionen zur Ausleihe und Rückgabe von Medien und den damit verbundenen Fristen werden außerdem bereitgestellt. Administratoren können andere Nutzerkonten editieren und löschen, hierfür steht ihnen eine Suchfunktion zur Verfügung. 
\end{flushleft}

\begin{flushleft}
\textbf{Authentifizierte Nutzer:} Nach einer Validierung der E-Mail-Adresse ist ein Nutzer im System registriert und kann sich jederzeit mit seinen Accountdaten anmelden, danach kann er sich verfügbare Exemplare von Medien zur Abholung markieren, hierzu steht ihm die im nächsten Paragraph beschriebene Suchfunktion zur Verfügung. Wird das Exemplar innerhalb einer gesetzten Frist abgeholt, gilt es als ausgeliehen und der authentifizierte Nutzer wird per E-Mail automatisch über den Ablauf der Rückgabefrist informiert. Ein angemeldeter Benutzer kann seinen Account selbstständig löschen und sich jederzeit abmelden.
\end{flushleft}

\begin{flushleft}
\textbf{Anonyme Nutzer:} In der Standartkonfiguration erlaubt das System Besuchern des Webspaces, neben der Registrierungsmöglichkeit, den Zugriff auf eine Suchfunktion sowie das Herunterladen/Lesen kostenfreier und öffentlich zugänglicher Medien. Diese Berechtigungen können von Administratoren auf die Registrierungsmöglichkeit reduziert werden. Die Suchfunktion beinhaltet eine graphische Darstellung der gesamten Kategoriehierarchie und die Möglichkeit sowohl nach Medien als auch nach deren Attributen zu suchen und diese Suche nach Kategorien zu filtern. Eine Detailansicht zu gefundenen Medien steht ebenfalls zur Verfügung.
\end{flushleft}

\subsection{Wunschkriterien}

\begin{flushleft}
\textbf{Administratoren:} Beim der Zuweisung der Kategorien für Medien ist die Hierarchietiefe unbeschränkt. Zusätzlich besitzen Administratoren die Möglichkeit, das System in einen manuellen Modus umzuschalten, in dem die Berechtigung zur Ausleihe eines Mediums von einem Administrator für einen authentifizierten Nutzer erst manuell freigeschaltet werden muss. Hierzu gibt es eine gesammelte Ansicht der noch nicht freigeschalteten Accounts.
\end{flushleft}

\begin{flushleft}
\textbf{Authentifizierte Nutzer:} Die Accountdaten des Nutzers können verändert und mit einem Avatarbild versehen werden, außerdem besteht die Möglichkeit die mit diesem Account ausgeliehenen/reservierten Medienexemplare einzusehen. Angemeldete Benutzer können alternativ zur Abholungsmarkierung eine Reservierung beantragen, sollten alle Exemplare eines Mediums ausgeliehen sein. Der Nutzer wird dann über den vorraussichtlichen Rückgabetermin des reservierten Exemplars informiert, und bei Abholungsmöglichkeit per E-Mail benachrichtigt. Er kann außerdem seine Reservierung zurückziehen.
\end{flushleft}

\begin{flushleft}
\textbf{Servicemitarbeiter:} Mit dieser Nutzerrolle haben Mitarbeiter der Bibliothek die Möglichkeit, zur Abholung markierte Exemplare gesammelt einzusehen und abzuarbeiten, indem sie den Abholstatus verändern. Außerdem können sie zurückgegebene Ausleihen wieder als verfügbar markieren. Zusätzlich ist es ihnen möglich Attributwerte für Medien und Exemplare zu editieren.
\end{flushleft}

\subsection{Abgrenzungskriterien}

Der Hauptfokus des Produkts liegt auf der Organisation und Verwaltung von Medien und Nutzern der Bibliothek, Funktionalitäten zum Erwerben und Verwalten von Lizenzen und/oder neuen Medien stehen nicht zur Verfügung. Die Abwicklung oder Verfolgung der Kundenzahlungen für Mitgliedsbeiträge, Medienerwerb o.ä. ist ebenfalls nicht im System integriert. Es besteht keine Möglichkeit, andere Bibliotheksmanagementsysteme oder Literaturenzyklopädien mit dieser Software zu verknüpfen. Das System ist für die klassische Benutzung über einen mit Tastatur, Maus und Bildschirm ausgestatteten PC/Laptop gedacht, mobile Geräte oder barrierefreie Benutzung werden begrenzt bis gar nicht unterstützt.

\newpage

\section{Produkteinsatz}

\newpage

\section{Produktumgebung}
\sectionauthor{Jonas Picker}

\subsection{Software}

Im folgenden Abschnitt werden Softwareabhängigkeiten und Kompatibilitäten der Software beschrieben.

\begin{itemize}
\item \underline{\textbf{Clientsoftware}}: \linebreak
Die Applikation wird über einen Webbrowser benutzt, dieser sollte die Darstellung in den Auszeichnungssprachen HTML, Version 5 und CSS, Level 3 unterstützen, sowie die Kommunikation mit dem HTTP/2-Protokoll. Explizit getestet werden die aktuellen Versionen der weit verbreiteten Browser: Google Chrome, Version: 88.0; Mozilla Firefox, Version: 85.0 und Apple Safari, Version: 14.0.3. Da HTML 5 (und CSS) jedoch seit einem guten Jahrzehnt als Web-Standart gilt, ist weitreichende Abwärts-kompatibilität bei den meisten Browsern zu erwarten.  

\item \underline{\textbf{Serversoftware}}: \linebreak
Auf dem Server muss die Laufzeitumgebung 'Java Virtual Maschine' verfügbar sein, zudem benötigt die Anwendung den Java Enterprise Applikationsserver Apache Tomcat, Version 10.0.x, dieser bringt zwar bei der Installation eine Java Laufzeitumgebung mit sich, es wird jedoch empfohlen, das komplette 'Development Kit' OpenJDK 16 herunterzuladen und bei Tomcat zu registrieren um Kompatibilitätsprobleme zu vermeiden. Außerdem muss ein E-Mail-Server bereit stehen, der das SMTP-Protokoll unterstützt. Die verwendete Datenbank muss das objektrelationale Datenbankmanagementsystem PostgreSQL, Version 13 benutzen, zusätzlich muss die Datenbank über ein gültiges SSL-Zertifikat verfügen um Datenübertragungen mit dem TLS-Protokoll zu ermöglichen. Die Installation der JVM/Tomcat, das Aufspielen der Anwendung sowie die Registrierung der Datenbank und des E-Mail-Servers werden in einem Installationsdokument beschrieben.
\end{itemize}

\subsection{Hardware}

Hier werden die geschätzen Rahmenbedingungen der Hardware für einen reibungslosen Betrieb aufgelistet, obwohl konkretere Messungen erst mit der Fertigstellung des Produkts möglich sind, skalieren die verwendeten Technologien, und damit die Kapazitäten der Anwendung, prinzipiell mit dem Aufstocken der Rechenleistung (am Server) mit. Sollten die Rahmenbedingungen unterschritten werden, ist ein Betrieb meistens immer noch möglich, jedoch nicht garantiert.

\begin{itemize}
\item \underline{\textbf{Clienthardware}}: \linebreak
Die Bedienung der Anwendung über die oben beschriebenen Browsertypen setzt voraus, dass Clientrechner die jeweiligen Hardwareanforderungen für deren Betrieb erfüllen.
\item \underline{\textbf{Serverhardware}}: \linebreak
Die Mindestanforderungen zum Betreiben einer Datenbank unter PostgreSQL bzw. eines Applikationsservers unter Tomcat reichen für einen sinnvollen Betrieb der Anwendung unter Last nicht aus, wie oben erwähnt sollten die der Anwendung zur Verfügung stehenden Ressourcen mit der Nutzerzahl in Relation stehen. Als Referenzplattform zum Betreiben der Anwendung dient der im Abschnitt 'Entwicklungsumgebung' beschriebene Rechner 'schratz' für Datenbank und Server gleichermaßen.
\end{itemize}

\subsection{Schnittstellen}

\begin{itemize}
\item \underline{\textbf{Clientenschnittstellen}}: \linebreak
Die notwendigen Mensch-Maschine Schnittstellen zum Benutzen der Anwendung über einen Browser sind: Bildschirm, Tastatur. Die Benutzung einer Maus oder vergleichbare Cursorsteuerung wird dringend empfohlen. Die Verifizierung eines neuen Benutzers bei der Registrierung im System wird über eine gültige E-Mail-Adresse abgewickelt, folglich muss der Nutzer Zugriff auf ein E-Mail-Konto besitzen. Der Browser des Klienten muss beim Benutzen über eine konstante Anbindung an das Internet verfügen, eine außergewöhnlich hohe Bandbreite ist nicht erforderlich.
\item \underline{\textbf{Serverschnittstellen}}: \linebreak
Der Server muss über notwendige Mensch-Maschine Schnittstellen zur initialen Installation der Anwendung und ihrer Softwareabhängigkeiten verfügen. Auch benötigt er eine konstante Anbindung an das (Intra- und) Internet, um Klientenanfragen beantworten, den E-Mail-Server kontaktieren und Datenbankanfragen absetzen zu können. Die Datenbank kann sich auf dem gleichen Gerät wie der Server, seperat davon im Netzwerk oder geographisch getrennt 'im Internet' befinden. Sie muss jedoch vom Server aus über einen Hostnamen ansteuerbar sein, je nach Realisierung ist dazu ein VPN-Tunnel notwendig. 
\end{itemize}

\newpage

\section{Produktfunktionen}

\newpage

\section{Produktleistung}

\newpage

\section{Benutzeroberfläche}

\newpage

\section{Qualitätsanforderungen}

\newpage

\section{Testing}

\newpage

\section{Entwicklungsumgebung}
\sectionauthor{Jonas Picker}

Die Entwicklung finden auf verschiedenen Privatrechnern der Teammitglieder statt, deren Bauteile und Betriebssysteme in den folgenden Listen kurz umrissen werden, keiner der Rechner weißt sonstige Besonderheiten auf. 

\subsection{Software}

\begin{itemize}
\item \underline{\textbf{Dokumentenbearbeitung}}: 
\begin{flushleft}
LaTeX Distribution MiKTeX, Version: 21.2
\end{flushleft}
\item \underline{\textbf{Webbrowser}}:
\begin{flushleft}
Google Chrome, Version: 88.0 \linebreak
Mozilla Firefox, Version: 85.0 \linebreak
Apple Safari, Version: 14.0.3 \linebreak
\end{flushleft}
\item \underline{\textbf{Integrierte Entwicklungsumgebungen}}: 
\begin{flushleft}
Eclipse IDE for Enterprise Java Developers, Version: 2020-12 (4.18.0) \linebreak
IntelliJ IDEA 2021.1 Ultimate Edition \linebreak
\end{flushleft}
\item \underline{\textbf{Objektorientierte Modellierung}}: 
\begin{flushleft}
IBM Rational Software Architect Designer 9.7 \linebreak
\end{flushleft}
\item \underline{\textbf{Programmiersprachen, Entwicklungs- \& Testframeworks}}: 
\begin{flushleft}
Java OpenJDK JDK 16 GA-Release\linebreak
Jakarta EE 9 \linebreak
Jakarta Server Faces 3.0 Mojarra Implementation \linebreak
CDI 3.0 with Red Hat Weld, Version: 4.0.1.Final \linebreak
Cascading Style Sheets Level 3 \linebreak
JUnit Platform, Version: 1.7.1 \linebreak
Selenium Server (Grid), Version: 3.141.59 \linebreak
\end{flushleft}
\item \underline{\textbf{Versionsmanagement}}:
\begin{flushleft}
git, Version 2.31.1 \linebreak
\end{flushleft}
\item \underline{\textbf{Betriebssysteme der Entwicklungsrechner}}:
\begin{flushleft}
MacOS Big Sur, Versionen: 11.2.3 und 11.2.1 \linebreak
Windows 10 Home 20H2 \linebreak
GNU/Linux Arch, Version: 5.11.11 \linebreak
GNU/Linux Debian, Version: 4.19.132-1 \linebreak
\end{flushleft}
\item \underline{\textbf{Graphische Prototypenbearbeitung}}:
\begin{flushleft}
Balsamiq Wireframes, Version: 4.2.4 \linebreak 
\end{flushleft}
\item \underline{\textbf{Applikationsserver}}: 
\begin{flushleft}
Apache Tomcat, Version: 10.0.2 \linebreak
\end{flushleft}
\item \underline{\textbf{Datenbanktreiber und Visualisierung}}: 
\begin{flushleft}
PostgreSQL JDBC 4.2 Driver, Version: 42.2.19 \linebreak
DBeaver, Version: 21.0.2
\end{flushleft}
\item \underline{\textbf{Teamkommunikation}}: 
\begin{flushleft}
Slack \linebreak
Discord \linebreak
Skype \linebreak
Stud.IP \linebreak
\end{flushleft}
\end{itemize}

\subsection{Hardware}

\begin{itemize}
\item \underline{\textbf{Entwicklungsrechner}}: 
\begin{flushleft}
MacBook Pro, 8GB RAM, Intel Core i5 2GHz  \linebreak
HP Laptop 15-db1xxx, 16GB RAM, AMD Ryzen 5 3500U 2.1GHz \linebreak
2x MacBook Pro, 8GB RAM, Intel Core i5 2.3GHz \linebreak
Tower-PC, 16GB RAM, AMD Ryzen 5 3600XT 3.8GHz \linebreak
\end{flushleft}
\item \underline{\textbf{Referenzrechner}}:
\begin{flushleft}
CIP-Pool Computer 'schratz', Uni Passau, 15.6GB RAM, Intel Core i7-4790 3.60GHz CPU, Intel I217-LM Ethernet Controller \linebreak
Spezifikation der virtualisierten Datenbank (siehe Entwicklungsschnittstellen) bezieht sich auf die Referenzhardware von 'schratz' und besitzt eine geschätze Übertragungsrate von ~1GB/s ins Internet. Die Datenbank unterstützt das Verschlüsselungsprotokoll TLS mit einem Zertifikat. \linebreak
\end{flushleft}
\end{itemize}

\subsection{Entwicklungsschnittstellen}

\begin{itemize}
\item \begin{flushleft} Netzwerk- und Internetverbindung \end{flushleft} 
\item \begin{flushleft} Virtuelle private Netwerkanbindung mit OpenVPN, Version 2.5.1 \end{flushleft} 
\item \begin{flushleft} Git-verwaltetes Repository der Uni Passau, Website: https://git.fim.uni-passau.de/ \end{flushleft} 
\item \begin{flushleft} Virtualisierte Datenbank der Uni Passau, Hostname: bueno.fim.uni-passau.de \end{flushleft}
\end{itemize}

\newpage

\section{Glossar}
\sectionauthor{Jonas Picker}

\end{document}
