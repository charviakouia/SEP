
\documentclass{article}

\usepackage[ngerman]{babel}
\usepackage{graphicx}
\usepackage{indentfirst}
\usepackage{hyperref}
\usepackage{geometry}
\usepackage{changepage}
\usepackage{booktabs}
\usepackage{float}
\usepackage{tabulary}
\usepackage{xcolor}
\usepackage{multirow}
\usepackage{subfig}

\graphicspath{ {./images/} }
\setlength\parindent{0pt}

\hypersetup{
    colorlinks,
    linkcolor={red!50!black},
    citecolor={blue!50!black},
    urlcolor={blue!80!black}
}

\makeatletter
\newcommand{\sectionauthor}[1]{
	{\parindent 0em \large \scshape Autor: #1 \par \nobreak \vspace*{1em}}
	\@afterheading
}
\newcommand{\specification}[3]{
	{\parindent 0.5em \hangindent 3em \hypertarget{spec:#1:#2}{\textbf{/#1#2/}} #3 \par \nobreak \vspace*{0.5em}}
}
\makeatother

\title{Bibliotheksanwendung - Feinspezifikation}
\date{\today\\v1.0}
\author{
	Ivan Charviakou\\
	León Liehr\\
	Mohamad Najjar\\
	Jonas Picker\\
	Sergei Pravdin
}

\begin{document}
\maketitle
\begin{figure}[h]
	\centering
	\includegraphics[width = 30em]{Logo}
\end{figure}
\newpage
\tableofcontents
\newpage

%----------------------------------------------------------------------Kapitel 1--------------------------------------------------------------------------------------------
\section{Einleitung}
Bei diesem Dokument handelt es sich um die Feinspezifikation unseres Bibliothekssystems. Es baut direkt auf dem vorangegangenen Entwurf auf und enthält einen noch genaueren Umriss der zu erstellenden Applikation.

%----------------------------------------------------------------------Kapitel 2--------------------------------------------------------------------------------------------


%----------------------------------------------------------------------Kapitel 3--------------------------------------------------------------------------------------------


%----------------------------------------------------------------------Kapitel 4--------------------------------------------------------------------------------------------


%----------------------------------------------------------------------Kapitel 5--------------------------------------------------------------------------------------------
\section{Systemkonfiguration}
Dieser Abschnitt beschreibt die Konfigurationsdatei und Sprachausgabe der Anwendung.
\subsection{Konfigurationsdatei}
\indent Die Anwendung wird in einer Datei namens 'config' im \hyperlink{https://de.wikipedia.org/wiki/Java-Properties-Datei}{{\texttt.properties}-Format} konfiguriert. Diese Datei wird dann mit Java-internen Funktionalitäten \hyperlink{https://docs.oracle.com/javase/7/docs/api/java/util/Properties.html}{dieser Klasse} über einen Stream eingelesen. Sie ist im Verzeichnis der Anwendung unter \texttt{/WebContent/WEB-INF/} zu finden und ist unten sowohl kurz \hyperlink{configTabelle}{beschrieben}, als auch \hyperlink{configFoto}{fotographiert} .
\subsubsection{Inhaltsbeschreibung}
\hypertarget{propSchema}{}
Die mit \textcolor{green}{!} oder \textcolor{green}{\#} eingeleiteten Zeilen der Dateien enthalten Kommentare. Alle unmarkierten Zeilen halten ein Paar aus einem Schlüssel und einem mit \texttt{:} getrennten Wert. Nur die Werte dürfen verändert werden! 

\begin{center}
\begin{table}[ht]
\begin{tabular} {| m{4cm} | m{6cm} | m{5cm} |}
\hline
\hypertarget{configTabelle}{Schlüsselname} & Wertebeschreibung & Wertebereich \\
\hline
DB\_HOST & Die URL (bzw. URI) unter der der Datenbankserver erreichbar ist.& Ein \hyperlink{https://datatracker.ietf.org/doc/html/rfc3986}{'Uniform Ressource Identifier'}. Pflichtfeld!\\
\hline
DB\_PORT & Der Port unter dem der Server Anfragen entgegennimmt. & Eine ganze Zahl von 0 bis 65535. Pflichtfeld!\\
\hline
DB\_NAME & Der Name der Datenbank auf dem Server. & Ein String. Pflichtfeld!\\
\hline
DB\_USER & Der Name eines in der Datenbank registrierten Benutzers. & Ein String. Pflichtfeld!\\
\hline
DB\_PASSWORD & Das Passwort des oben angegebenen Datenbankbenutzers. & Ein String. Pflichtfeld!\\
\hline
DB\_CAPACITY & Die maximale Anzahl an Verbindungen, die die Datenbank dem Server zu Verfügung stellen kann. & Eine natürliche Zahl (nicht 0). Standartwert ist 42.\\
\hline
MAILSERVER\_HOST & Der URI unter der der E-Mail-Server erreichbar ist. &  Ein \hyperlink{https://datatracker.ietf.org/doc/html/rfc3986}{'Uniform Ressource Identifier'}. Pflichtfeld! \\ 
\hline
MAILSERVER\_PORT & Der Port durch den Anfragen an den Server geleitet werden. & Eine ganze Zahl von 0 bis 65535. Pflichtfeld! Standartwert ist 25. \\
\hline
MAIL\_SOURCE & Die Absenderadresse für E-Mails des Bibliothekssystems. & Gültige E-Mail-Adressen folgen \hyperlink{https://datatracker.ietf.org/doc/html/rfc5322}{diesem Schema}. Pflichtfeld!\\
\hline
SCAN\_INTERVAL & Der Wartungsthread des Systems führt in diesem Interval automatische Aufgaben durch. & Eine natürliche Zahl (nicht 0) die Minuten repräsentiert. Standartwert ist 20. \\
\hline
DEFAULT\_ADMIN\_MAIL & Diese Zeichenkette wird ohne Verifizierung als E-Mail des ersten Administratoraccounts eingetragen, falls es noch keinen gibt (im System auch änderbar).  & Gültige E-Mail-Adressen folgen \hyperlink{https://datatracker.ietf.org/doc/html/rfc5322}{diesem Schema}, andernfalls kommen versendete Mails nicht an. Schlüssel ist auch Standartwert.\\
\hline
LOG\_LEVEL & Eine Granularitätsoption für die Protokollierung der Fehler und Systemabläufe. & Im Detailgrad aufsteigend: 'SEVERE', 'DETAILED', 'DEVELOPMENT'. Standart ist 'SEVERE', andere Werte werden ignoriert. \\
\hline
LOG\_CONSOLE & Gibt an, ob die Systemprotokolle auch auf der Konsole ausgegeben oder nur ins Log geschrieben werden. & Ein Boolean aus 'TRUE' oder 'FALSE'. Andere Werte werden ignoriert. Standartwert ist 'FALSE'. \\
\hline
\end{tabular}
\end{table}
\end{center}

\newgeometry{left=0cm,right=0cm,top=0cm,bottom=0cm}
\begin{figure}
\hypertarget{configFoto}{}
\centering
\includegraphics[angle = 270, width = 53em]{configProperties}
\caption{Die \texttt{config.properties} Datei}
\end{figure}
\restoregeometry

\subsection{Sprachressourcen}
\indent Obwohl die Anwendung primär in Deutsch ausgeliefert wird, bietet die modulare Bauweise eine Erweiterungsmöglichkeit um andere Sprachausgaben. Dazu werden alle angezeigten Strings einer Sprache zentral im Ordner \texttt{/src/ressources/} des Applikationsverzeichnisses im \hyperlink{https://de.wikipedia.org/wiki/Java-Properties-Datei}{\texttt{.properties}-Format} gespeichert. Die Namenskonvention für diese Dateien ist \texttt{messages\_xx.properties}, wobei \texttt{xx} durch das jeweilige \hyperlink{https://de.wikipedia.org/wiki/Liste_der_ISO-639-1-Codes}{Sprachenkürzel} ersetzt wird. Die Pakete sind durch Eintrag in die \texttt{faces-config.xml} im JSF-Framework registriert. Damit sind die Schlüssel mittels EL in den Facelets referenzierbar. So kann die Sprache, entweder auf der Profilseite manuell oder auch automatisch durch die Browserpräferenz, pro Nutzer-Session dynamisch eingestellt werden. Standartsprache ist Deutsch. Die Dateien selbst folgen obigen \hyperlink{propSchema}{Schema}. Die Schlüssel können einfach kopiert, mit Werten ergänzt und in den \texttt{/ressources}-Ordner gelegt werden. Ein Beispiel für die \hyperlink{messagesde}{deutsche} und \hyperlink{messagesen}{englische} Datei ist anbei.

\newgeometry{left=0cm,right=0cm,top=0cm,bottom=0cm}
\begin{figure}
\hypertarget{messagesen}{}
\centering
\includegraphics[width=60em]{messagesen}
\caption{Englisches Sprachpaket}
\end{figure}

\begin{figure}  
\hypertarget{messagesde}{}
\centering
\includegraphics[width=60em]{messagesde}
\caption{Deutsches Sprachpaket}
\end{figure}
\restoregeometry

%----------------------------------------------------------------------Kapitel 6--------------------------------------------------------------------------------------------


%----------------------------------------------------------------------Kapitel 7--------------------------------------------------------------------------------------------
\section{Datenbankschema}
In diesem Abschnitt wird die Umwandlung des ER-Diagramms (Siehe Entwurf Kapitel 7) in ein relationales Datenbankschema behandelt und untersucht.
\subsection{Umwandlungsentscheidungen}
\textbf{Starke Entitäten:} Die starken Entitäten \hyperlink{Medium}{'Medium'},\hyperlink{Category}{'Kategorie'} und \hyperlink{User}{'Benutzer'} werden klassisch in eigene Tabellen mit den markierten Primärschlüsseln umgewandelt. \hyperlink{Application}{'Anwendung'} wird als einzeilige Tabelle mit konstantem Primärschlüssel umgesetzt. Die Spezialisierungen der Entität 'Benutzer' werden, aus Mangel an eigenen Attributen, in die Benutzertabelle mittels des neuen Attributs 'userRole' integriert. \\
\textbf{Schwache Entitäten:} Die zum Benutzer gehörige Entität 'Adresse' wird mit der \hyperlink{User}{Benutzertabelle} fusioniert, da wir die Adresse in unserem System nicht weiterverarbeiten und die Anzahl zusammenwohnender Nutzer verschwindend gering sein wird. Da es sich bei den Attributen 'Signatur' und 'Bibliotheksstandort' um nutzergenerierte Werte handelt, wird die vom Medium abhängige 'Exemplar'-Entität in eine \hyperlink{Copy}{Tabelle} mit dem Kompositschlüssel (\textit{mediumID, copyID}) transformiert. Mit den Entitäten \hyperlink{CustomAttribute}{'Attribut'} und \hyperlink{AttributeType}{'Attributtyp'} wird ebenso verfahren, die jeweiligen Kompositschlüssel sind (\textit{mediumID, attributeID}) und (\textit{attributeID, typeID}). \\
\textbf{Relationen:} Alle hat-ein-Relationen verschwinden. Da sich die Relationen 'markiert' und 'ausgeliehen' gegenseitig ausschließen, werden sie in der \hyperlink{Copy}{'Exemplar'-Tabelle} in einen einzigen Fremdschlüssel umgewandelt, der zusammen mit dem Verfügbarkeitsstatus die gleichen Funktionalitäten der Originalrelationen abbilden kann. Aus der Relation 'ist Kind von' entsteht ebenfalls eine eigene Spalte \textit{parent} in der \hyperlink{Category}{'Kategorie'-Tabelle}. Da der oberste Knoten der Hierarchie keinen Elternknoten besitzt, ist hier ebenfalls \texttt{NULL} erlaubt. Bei Löschung der Elternkategorie wird eine kaskadierende Löschung der Kinder vorgenommen. Die Relation 'eingeordnet unter' wird in einen Fremdschlüssel in der \hyperlink{Medium}{'Medium'-Tabelle} umgewandelt, der auch \texttt{NULL} als Wert erlaubt. Hier wird von den im ER-Diagramm spezifizierten Kardinalitäten abgewichen und Medien, die nicht manuell in eine Kategorie eingeordnet wurden, nehmen nicht an der Relation Teil (\texttt{NULL} repräsentiert den Top-Knoten). \\
\subsection{Konventionen und eigene Datentypen}
\textbf{Namenskonventionen:} Eigens definierte Datentypen und Attribute normaler Entitätstabellen werden in Camel-Case, aber klein geschrieben. Werte der \texttt{ENUM}s werden klein geschrieben. Tabellennamen sind in normalem Camel-Case. \\
\textbf{Eigene Datentypen:} Statt dem Einlesen von Strings werden im Java-Code des Systems Enums verwendet, um Zustände zu speichern. Diese werden auch in der Datenbank in Form von eigenen \hyperlink{ENUMs}{Enumerationstypen} gespeichert. 
\newgeometry{left=0cm,right=0cm,top=0cm,bottom=0cm}
\begin{figure}
    \hypertarget{ENUMs}{}
    \centering
    \subfloat{{\includegraphics[width=9cm]{SQLENUMs1} }}
    \qquad
    \subfloat{{\includegraphics[width=9cm]{SQLENUMs2} }}
    \caption{Die systemeigenen \texttt{ENUM}-Typen}
\end{figure}


\begin{figure}
\hypertarget{User}{}
\flushleft
\includegraphics[width=50em]{SQLUSER}
\caption{SQL-Anweisung zum Erstellen der Nutzertabelle, mit automatischem Setzen der Tokenerstellzeit.}
\end{figure}

\begin{figure}
\hypertarget{Medium}{}
\flushleft
\includegraphics[width=60em]{SQLMEDIUM}
\caption{SQL-Anweisung zum Erstellen der 'Medium'-Tabelle.}
\end{figure}

\begin{figure}
\hypertarget{Application}{}
\flushleft
\includegraphics[width=45em]{SQLAPPLICATION}
\caption{SQL-Anweisung zum Erstellen der 'Applikations'-Tabelle. Hier werden Anwendungseinstellungen gespeichert.}
\end{figure}

\begin{figure}
\hypertarget{Category}{}
\flushleft
\includegraphics[width=60em]{SQLCATEGORY}
\caption{SQL-Anweisung zum Erstellen der 'Kategorie'-Tabelle.}
\end{figure}

\begin{figure}
\hypertarget{CustomAttribut}{}
\flushleft
\includegraphics[width=60em]{SQLCUSTOMATTRIBUTE}
\caption{SQL-Anweisung zum Erstellen der 'Attribute'-Tabelle.}
\end{figure}

\begin{figure}
\hypertarget{AttributeType}{}
\flushleft
\includegraphics[width=60em]{SQLATTRIBUTETYPE}
\caption{SQL-Anweisung zum Erstellen der Tabelle für Attributtypen.}
\end{figure}

\begin{figure}
\hypertarget{Copy}{}
\centering
\includegraphics[width=60em]{SQLCOPY}
\caption{SQL-Anweisung zum Erstellen der 'Exemplar'-Tabelle mit Integritätschecks für die 'ausgeliehen'- und  'markiert'-Relationen.}
\end{figure}
\restoregeometry

\subsection{Normalformanalyse}
Zunächst werden die erstellten Tabellen in klassischer Notation für Relationen aufgelistet: \\
\\
\textbf{User}:\{ \textit{userID, emailAddress, passwordHashSalt, passwordHash, name, surname, postalCode, city, street, houseNumber, token, tokenCreation, userLendPeriod, lendStatus, verificationStatus,\\ userRole} \}\\
\textbf{Medium}:\{ \textit{mediumID, hasCategory} \}\\
\textbf{Application}:\{ \textit{one, bibName, emailRegEx, contactInfo, imprintInfo, privacyPolicy, bibLogo, globalLendLimit, globalMarkingLimit, reminderOffset, registrationStatus, lookAndFeel, anonRights,\\ userLendStatus} \}\\
\textbf{Category}:\{ \textit{categoryID, title, description, parent} \}\\
\textbf{CustomAttribute}:\{ \textit{attributeID, fromMedium, attributeName, attributeValue} \}\\
\textbf{AttributeType}:\{ \textit{typeID, fromAttribute, previewPosition, multiplicity, modifiability, dataType} \}\\
\textbf{Copy}:\{ \textit{copyID, fromMedium, signature, bibPosition, status, deadline, actor} \}\\
\\
Aus den funktionalen Abhängigkeiten zwischen den Attributen ergeben sich verschiedene Superschlüssel für die Relationen. Die Superschlüsselmöglichkeiten sind eingeklammert.\\
\\
\textbf{User}:\{ (\texttt{userID}), (\texttt{emailAddress}), (\texttt{token}), (\texttt{userID, token}), (\texttt{userID, emailAddress}), \\ (\texttt{emailAddress, token}), (\texttt{userID, emailAddress, token}), \textit{passwordHashSalt, passwordHash, name, surname, postalCode, city, street, houseNumber, tokenCreation, userLendPeriod, lendStatus, verificationStatus, userRole} \}
Es gibt sowohl Städte mit mehreren Postleitzahlen, als auch Postleitzahlen, die mehrere Städte/Dörfer einschließen. Somit legen sich die Attributwerte nicht gegenseitig eindeutig fest und es besteht keine funktionale Abhängigkeit.\\
\textbf{Medium}:\{ (\texttt{mediumID}), \textit{hasCategory} \}\\
\textbf{Application}:\{ (\texttt{one}),\textit{ bibName, emailRegEx, contactInfo, imprintInfo, privacyPolicy, bibLogo, globalLendLimit, globalMarkingLimit, reminderOffset, registrationStatus, lookAndFeel, anonRights,\\ userLendStatus} \}\\
\textbf{Category}:\{ (\texttt{categoryID}), \textit{title, description, parent} \}\\
\textbf{CustomAttribute}:\{ (\texttt{fromMedium, attributeID}),  \textit{attributeValue, attributeName} \}\\
\textbf{AttributeType}:\{ (\texttt{fromAttribute, typeID}), \textit{previewPosition, multiplicity, modifiability, dataType} \}\\
\textbf{Copy}:\{ (\texttt{fromMedium, copyID}), (\texttt{fromMedium, signature}), (\texttt{fromMedium, signature, copyID}), \textit{bibPosition, status, deadline, actor} \}
Die Attribute \textit{status, deadline, actor} beeinflussen sich gegenseitig, \textit{deadline} hängt jedoch noch von anderen Limits außerhalb dieser Relation ab und wird somit nicht eindeutig durch Änderungen von \textit{actor} oder \textit{status} festgelegt. Da nur der Status 'available' den Wert \texttt{NULL} im Attribut \textit{actor} impliziert und alle anderen Werte keine eindeutige Zuordnung festlegen, gibt es hier keine schlüsselunabhängigen Abhängigkeiten.\\
\\
Aus den minimalen Superschlüssel ergeben sich die Kandidatenschlüsselmöglichkeiten. Da alle Attributewerte nur atomare Werte annehmen können, ist das Schema in erster Normalform. Da es ebenfalls kein nichtprimes\footnote{Ein Attribut, welches nicht Teil eines Kandidatenschlüssels ist.} Attribut gibt, welches von einer echten Teilmenge eines Schlüssel funktional abhängig ist, gilt für das Schema die zweite Normalform. Für jede funktionale Abhängigkeit \textit{X $\rightarrow$ A} gilt außerdem: Entweder ist \textit{A} prim oder\textit{X} ist ein Superschlüssel der Relation. Damit folgt die dritte Normalform. Ohne schlüsselunabhängige Abhängigkeiten bleiben zur Untersuchung nur noch Attribute der überlappenden Kandidatenschlüssel übrig. In keine Richtung bestehen weder zwischen \textit{userID $\leftrightarrow$ token}, \textit{emailAddress $\leftrightarrow$ token} oder \textit{emailAddress $\leftrightarrow$ userID} noch zwischen \textit{signature $\leftrightarrow$ copyID} funktionale Zusammenhänge. Das Relationenschema ist somit in Boyce-Codd-Normalform, da alle Abhängigkeiten von Superschlüsseln ausgehen.



%----------------------------------------------------------------------Kapitel 8--------------------------------------------------------------------------------------------


%----------------------------------------------------------------------Kapitel 9--------------------------------------------------------------------------------------------
\section{Sicherheit}
Dieses Kapitel baut auf dem Abschnitt 'technische Systemsicherheit' des Entwurfs auf und vertieft Sicherheitsaspekte der Anwendung.
\subsection{Injections}
\hypertarget{Injections}{Das} direkte Interpretieren von nutzergenerierten Eingaben wird unterbunden.\\
\textbf{SQL:} Da unser System postgreSQL als Datenbankanfragesprache verwendet, wird in der \hyperlink{DAOs}{Datenzugriffsschicht}!!! im Java-Code mit \hyperlink{https://docs.oracle.com/javase/7/docs/api/java/sql/PreparedStatement.html}{'PreparedStatements'} gearbeitet. Da die eigentliche Anfrage nicht zusammen mit den vom Nutzer stammenden Parametern interpretiert wird, hat das Einschleusen von SQL-Code in diese Parameter keinen Effekt auf die Datenbank.\\
\textbf{HTML/JavaScript:} \hypertarget{XSS}{Um} das Einschleusen von HTML- (und eingebetteten JavaScript-) Code, in nutzergenerierte Inhalte der Webseite zu verhindern, verfügen die JSF-Facelet Elemente für User-Inputs über integrierte Escaping-Funktionen. Alle nutzergenerierten Anzeigen der Webseite werden so direkt am Input geschützt.
\subsection{Insecure Direct Object References}
Unbefugte direkte Zugriffe auf zugangsbeschränkte Teile der Anwendung werden abgefangen.\\ 
Nutzer könnten versuchen, manuell die URL-Parameter in der Adresszeile zu verändern. In der Medienansicht könnte man so z.B. durch Ändern der ID auf die Ansicht eines anderen Mediums gelangen. Bei einer Anfrage werden erste Maßnahmen der Zugriffskontrolle nach der \texttt{RESTORE\_VIEW}-Phase im \hyperlink{PhaseListener}{Trespasslistener}!!! vorgenommen. Hier wird über die Sessionmap die Nutzerrolle und der Loginstatus mit der URL-Struktur der Anfrage abgeglichen und bei invalidem Zugriff entsprechend umgeleitet. Um denselben Zugriffsprozess wie im Mediumsansicht-Beispiel bei der Profilseite zu verhindern, wird im Backing-Bean durch Zugriff auf die URL-Parameter geprüft, ob der Nutzer berechtigt ist, andere Profile als sein eigenes anzusehen. 
\subsection{Session Fixation}
Wie bereits im Abschnitt 'Session-Hijacking' des Entwurfs erwähnt, tauschen wir unmittelbar nach dem Login eines Benutzers im \hyperlink{Login}{Backing-Bean}!!! den Identifikator der Nutzer-Session aus. \hyperlink{Session}{UserSession}!!!-Objekte werden nur für eingeloggte Nutzer angelegt und beim Login werden keine vorherigen Zustandsinformationen übernommen.  Zusätzlich zur Zerstörung beim Logout haben alle Sessions Tomcat-intern einen Inaktivitätstimeout von 30 Minuten, den wir beibehalten werden (andere Servlet Container verwenden denselben Timeout).
\subsection{Userinput}
Neben dem Vorbeugen von \hyperlink{Injections}{Injections} gibt es weitere Maßnahmen zum Absichern von Nutzereingaben.\\
\textbf{Admins:} Ausnahmen von den \hyperlink{XSS}{XSS-Gegenmaßnahmen} bilden das Impressum, die Datenschutzerklärung und  die Kontaktinformationen, da wir Administratoren beim editieren mehr Gestaltungsmöglichkeiten geben wollen. Andere Nutzerrollen haben keinen schreibenden Zugriff auf diese Teile der Webseite.\\
\textbf{Passwörter:} Eine Mindestlänge von 8 Zeichen in Groß- und Kleinschreibung und das enthalten einer Zahl wird durch den \hyperlink{Passwordvalidator}{PasswordValidator}!!! sichergestellt. Registrierte Passwörter werden nicht im Klartext gespeichert, sondern in der Klasse \hyperlink{Hash}{PasswordHashingModule}!!! durch eine Hashfunktion geschickt. Durch vorheriges Anhängen eines Salts\footnote{Ein kurzer, zufällig generierter String.} an das Passwort werden Kollisionen der Hashfunktion unwahrscheinlicher und gleiche Passwörter entsprechen nicht mehr demselben Hashwert.\\
\textbf{E-Mail-Adressen:} Bei der Registrierung eines neuen Nutzerkontos oder der Änderung der Adresse wird der vom Administrator festgelegte RegEx zuerst bedient und danach wird auf Duplikate in der Datenbank geprüft. Dies geschieht im \hyperlink{Emailvalidator}{EmailValidator}!!!. Der anschließende Verifizierungsprozess stellt die Gültigkeit der Adresse sicher.
\subsection{System Secrecy}
Die Hintergrundprozesse und Technologien werden bestmöglich versteckt.\\
\textbf{Rollenisolation:} Ist der Nutzer berechtigt, das unter der URL erreichbare Facelet zu sehen, werden, anhand der Nutzerrolle,vorhandene Knöpfe und Eingabefelder je nach Rollenberechtigungen gerendert oder ausgeblendet. So haben Rollen der unteren Hierarchieschichten keine Information über die Möglichkeiten der darüber liegenden.\\
\textbf{Technologiemaskierung:} Der XML-Dialekt der Facelet-Sprache ist für den Benutzer undurchsichtig, die Sprache wird durch das Framework in HTML und JavaScript mit der Endung '.xhtml' umgewandelt. Weder das verwendete Datenbanksystem noch das JSF-Framework sind für den Benutzer ersichtlich. Im laufenden System werden bei Fehlermeldungen eigene Fehlerseiten erstellt und keine Stack-Traces ausgegeben, um die Tomcat-eigenen Fehlerseiten nicht preiszugeben. Wenn möglich wird, durch die vorgezogene Verwendung von \textit{Forwards} statt \textit{Redirects} bei der Seitennavigation die URL gleich bleiben und dadurch die interne Ordnerstruktur zumindest teilweise maskiert.

\end{document}